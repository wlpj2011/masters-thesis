\chapter{Local Converse Theorem for Smooth Representations of the Weil Group}	%Chapter title
\begin{itemize}
\item Definition of Local field
\item unique maximal unramified extension and unique sub extensions of each degree prime to characteristic
\item Definition of Frobenius
\item Definition of Weil Group 
\item Definition of Smooth Representations of Weil Group; and semi-simple
\item Definition of Inertia subgroup 
\item Explain unramified/tamely ramified/highly ramified and inertia/wild-inertia and level of characters.
\item Explain what is necessary of additive characters and how we can choose a level 1?0? additive character and everything changes predictably if we change it
\item Definitions of $L$ functions and $\epsilon$ factors for characters of $F^\times$
\item Brief highlights for local class field theory
\item Definition of L function of representations of Weil group
\item Definition of $\epsilon$ factor of representation 
\item Definition of $\gamma$ factor of representation 
\item Definition of $\epsilon_0$ factor of representation 
\item Relate this to Gauss sums/what are the gauss sums the naturally appear in the $\epsilon$ factor? and what can they be reduced to?
\item Stability Theorem for characters of $F^\times$ from BH 23.8
\item 1D Local Converse Theorem
\begin{itemize}
\item $\gamma$ has $1$ pole; so dealing with an unramified character uniquely defined by it's pole/zero
\item $\gamma$ has no poles; use stability and twist by highly ramified characters 
\end{itemize}
\item 2D Local Converse Theorem 
\begin{itemize}
\item $\gamma$ has $2$ poles, 
\begin{itemize} 
\item $\rho$ is the sum of two unramified characters
\end{itemize}
\item $\gamma$ has $1$ pole
\begin{itemize}
\item $\rho$ is the sum of an unramified character and a tamely ramified character
\item $\rho$ is the sum of two unramified characters, one of which is $1/q$ times the other cancelling out a pole and a zero in $\gamma$
\end{itemize}
\item $\gamma$ has no poles:
\begin{itemize}
\item $\rho$ is the sum of two tamely ramified characters; then twisting by a level $1$ character gives $|\epsilon| \sim q^{1-2s}$. We have $\chi_1 \oplus \chi_2$ and $\chi_1' \oplus \chi_2'$. Twist both by $\chi_1^{-1}$ to force an unramified character, the poles of $\gamma$ must match, so either $\chi_1^{-1} \otimes \chi_1'$ or $\chi_1^{-1} \otimes \chi_2'$ must be unramified and infact equal to identity. Then we only have one character and can use stability. If we end up with $\chi_1 = \chi_2$ we get two poles after twisting and same on the other side.
\item $\rho$ is a $2$ dimensional tamely ramified irreducible induced by a tamely ramified character of an unramified quadratic extension; then twisting by a level $1$ character gives $|\epsilon| \sim q^{\frac{1}{2}-s}$. In this case, both characters must be induced by a tamely ramified character of the same unramified quadratic extension, so we can deal with the character instead of the dim $2$ irreducible. There we just use stability to prove the characters are the same. There will need to be a discussion of the Langlands constant and induction of $\epsilon$ here.
\end{itemize}
\end{itemize}
\item Do we ever actually need the finite field local converse theorem? 
\item Can we use finite field local converse theorem instead of stability?
\item Why does this proof technique break down eventually?
\item How far can it get? Can it prove $n = 3,4,5?$ Is $n=6$ when you start needing dim $2$ twists?
\item Can you get highly ramified $n=2$? Stability doesn't work as well here because of the level restrictions. What changes with non-semisimple $n=2$?
\end{itemize}

\section{Local Fields}
We are interested in a local converse theorem for smooth representations of the Weil group of a local field.We are interested in a local converse theorem for smooth representations of the Weil group of a local field.
Most of the material on local fields will echo that in Jean-Pierre Serre's \textit{Local Fields} \cite{Serre1979}, with the material on local class field theory coming from from Bushnell and Henniart's \textit{The Local Langlands Conjecture for $GL(2)$} \cite{Bushnell2006}.
Adopting the notation in the second of these references, we will have the following notation for local fields.\\
$F$ refers to a non-Archimedean local field.\\
$\fo$ is the discrete valuation ring in $F$.\\
$\fp$ is the unique maximal ideal of $\fo$.\\
$k = \fo/\fp$ is the residue field of $F$.\\
$q = p^n = |k|$ is the size of the residues field where $p$ is the characteristic.\\
$U_F$ is the group of units of $\fo$.\\
$U_F^n = 1 + \fp^n$ for $n \geq 1$ are subgroups of the unit group forming a filtration.\\
\begin{defn}
  A \textbf{discrete valuation ring} is a principal ideal domain $\fo$ with a unique non-zero prime ideal $\fp$.
\end{defn}
We also call such a ring $\fo$ a DVR.
An alternative characterization of a DVR is that it is a domain $\fo$ such that its field of fractions $K$ has a non-trivial valuation $v : K \to \bbZ \cup \{ \infty \}$; that is, a function $v$ such that $v(xy) = v(x) + v(y)$, $v(x+y) \geq \min \{v(x),v(y)\}$, $v(x) = 0$ if and only if $x = 0$, and $v$ takes more values than just $0$ and $\infty$.
\begin{defn}
  A \textbf{local field} is a field $F$ with a valuation $v$ that is locally compact with respect to the topology provided by the valuation and has finite residue field $k$ with $|k| = q = p^n$.
  The valuation defines a basis of open sets as the additive cosets of the $v^{-1}(\{x \in F \mid q^{-v(x)} \leq r\})$ for positive real numbers $r$.
\end{defn}
Given a local field $F$, it is not difficult to show that $\fo = \{x \in F \mid q^{-v(x)} \leq 1\}$ is the ring of integers of $F$ and is a DVR.
Further, $\fp = \{x \in F \mid q^{-v(x)} < 1\}$ is unique maximal (prime) ideal of the DVR $\fo$.
Finally, $\fo^\times = U_F = \{x \in F \mid q^{-v(x)} = 1\} = \{x \in F \mid v(x) = 0\}$ is the unit group of $\fo$.
A local field is isomorphic as a topological field to $\bbR$, $\bbC$, a finite extension of the $p$-adic numbers $\bbQ_p$, or the field of formal Laurent series $\bbF_q((T))$ over a finite field.
$\bbR$ and $\bbC$ are Archimedean local fields, so extensions of $\bbQ_p$ and $\bbF_q((T))$ will be the examples of the non-Archimdean local fields under consideration.
\\

For a DVR $\fo$ or for a local field $F$, we define a special element called the uniformizer, denoted by $\varpi$, which is a prime in $\fo$, so that $\varpi \fo = \fp$ is the unique prime ideal.
This choise is unique up to units.
\section{$L$ Functions and $\epsilon$ Factors of Local Fields}
Though we are interested in the representation theory of the Weil group, and how we can tell apart the representations of the Weil group, in order to define the objects we will use to do so, we must first define them for local fields.
Generally, a local converse theorem for local fields would deal with representations of $GL_n(F)$; however, all of the tools we need are developed while studying $GL_1(F) = F^\times$.\\

An additive character of a local field $F$ is a continuous homomorphism $\psi: F \to \bbC^\times$; these are the $1$ dimensional representations of the additive group of $F$.
Equivalently to saying $\psi$ is continuous, $\ker \psi$ is open in the topology defined by the valuation.
We let $\hat{F}$ be the group of additive characters of $F$, which are a group under pointwise multiplication.
\begin{defn}
  Let $\psi \in \hat{F}$ with $\psi \neq 1$.
  The \textbf{level} of $\psi$ is the smallest integer $d$ such that $\fp^d \subseteq \ker \psi$.
\end{defn}
We then have the following proposition mostly characterising the additive characters of $F$.
\begin{prop}
  Let $\psi \in \hat{F}$ with $\psi \neq 1$ be a level $d$ character.
  \begin{enumerate}
    \item Let $a \in F$. The map $a \psi : x \mapsto \psi(ax)$ is a character of $F$. If $a \neq 0$, then the character $a \psi$ has level $d - v_F(a)$, where $v_F$ is the valuation on $F$.
    \item The map $a \mapsto a \psi$ is a group isomorphism $F \cong \hat{F}$.
  \end{enumerate}
\end{prop}

We also define multiplicative characters of $F$ which are the $1$ dimensional representations of the multiplicative group of $F$.
In this setting, a multiplicative character of a local field $F$ is a continuous homomorphism $\chi : F^\times \to \bbC^\times$.
We also define level for multiplicative characters with a slight modification.
\begin{defn}
  Let $\chi$ be a non-trivial character of $F^\times$.
  The \textbf{level} of $\chi$ is defined to be the smallest integer $n \geq 0$ such that $U^{n+1}_F \subseteq \ker \chi$.
  We further say that $\chi$ is \textbf{unramified} if $U_F \subseteq \ker \chi$.
\end{defn}


The final thing we need to define the $L$ functions and $\epsilon$ factors for representations of $F^\times$ is a measure that we can define an integral with respect to.
\textcolor{red}{do we actually need this?}


We define \textcolor{red}{?} the $L$ function for a character of $F^\times$ as a variable of a complex variable $s$ by
\[L(\chi, s) = \begin{cases} (1 - \chi(\varpi) q^{-s})^{-1} & \text{ if } \chi \text{ is unramified,} \\ 1 & \text{ otherwise} \end{cases}.\]
This will be independent of the choice of uniformizer because unramified characters $\chi$ are trivial on units.
\\

\textcolor{red}{What about $\gamma$? Defining $\gamma$ requires defining the $\zeta$ integrals which requires Fourier transforms; measures, etc... and need to define $\check{\chi}$ either way}

We define the $\epsilon$ factor as
\[\epsilon(\chi, s, \psi) = \gamma(\chi, s, \psi) \frac{L(\chi,s)}{L(\check{\chi},1-s)}\]
where $\gamma$ is the unique rational function defining a certain equation\textcolor{red}{...}
For unramified characters $\chi$ and level one characters $\psi$, we find that
\[\epsilon(\chi, s, \psi) = q^{s - \frac{1}{2}} \chi(\varpi)^{-1}.\]

\begin{theorem}
  For characters $\chi$ of $F^\times$ with level $n \geq 0$ and not unramified, and $\psi \in \hat{F}$ with level one, we have that
  \[\epsilon(\chi, s, \psi) = q^{n(\frac{1}{2} - s)} \sum_{x \in U_F/U_F^{n+1}} \chi(\alpha x)^{-1} \psi(\alpha x)/q^{(n+1)/2}\]
  for any $\alpha \in F^\times$ such that $v(\alpha) = -n$.
\end{theorem}

Finally, we will need a theorem that describes how $\epsilon$ factors change as we twist them by high level characters.
\label{thm:Stability}
\begin{theorem}[Stability Theorem]
  Let $\theta, \chi$ be characters of $F^\times$ of level $ l \geq 0$ and $n \geq 1$.
  Suppose that $2l < n$.
  Let $\psi \in \hat{F}$ with $\psi \neq 1$ and let $c \in F$ satisfy $\chi(1+x) = \psi(cx)$ for $x \in \fp^{\lfloor n/2 \rfloor + 1}$.
  Then
  \[ \epsilon(\theta \chi, s, \psi) = \theta(c)^{-1} \epsilon(\chi, s, \psi).\]
\end{theorem}
\section{Weil Groups of Local Fields}

\section{Local Class Field Theory}

\section{$L$ Functions and $\epsilon$ Factors of Weil Groups}

\section{Local Converse Theorem for Local Fields}

\endinput
