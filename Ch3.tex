\chapter{Local Converse Theorem for Smooth Representations of the Weil Group}	%Chapter title
\begin{itemize}
\item Definition of Local field
\item Definition of Frobenius
\item Definition of Weil Group 
\item Definition of Smooth Representations of Weil Group; and semi-simple
\item Definition of Inertia subgroup 
\item Explain unramified/tamely ramified/highly ramified and inertia/wild-inertia and level of characters.
\item Explain what is necessary of additive characters and how we can choose a level 1?0? additive character and everything changes predictably if we change it
\item Definition of L function of representation 
\item Definition of $\epsilon$ factor of representation 
\item Definition of $\gamma$ factor of representation 
\item Definition of $\epsilon_0$ factor of representation 
\item Relate this to Gauss sums
\item Stability Thoerem for characters of $\mathbb{F}^\times$ from BH 23.8
\item 1D Local Converse Theorem
\begin{itemize}
\item $\gamma$ has $1$ pole; so dealing with an unramified character uniquely defined by it's pole/zero
\item $\gamma$ has no poles; use stability and twist by highly ramified characters 
\end{itemize}
\item 2D Local Converse Theorem 
\begin{itemize}
\item $\gamma$ has $2$ poles, 
\begin{itemize} 
\item $\rho$ is the sum of two unramified characters
\end{itemize}
\item $\gamma$ has $1$ pole
\begin{itemize}
\item $\rho$ is the sum of an unramified character and a tamely ramified character
\item $\rho$ is the sum of two unramified characters, one of which is $1/q$ times the other cancelling out a pole and a zero in $\gamma$
\end{itemize}
\item $\gamma$ has no poles:
\begin{itemize}
\item $\rho$ is the sum of two tamely ramified characters; then twisting by a level $1$ character gives $|\epsilon| \sim q^{1-2s}$. We have $\chi_1 \oplus \chi_2$ and $\chi_1' \oplus \chi_2'$. Twist both by $\chi_1^{-1}$ to force an unramified character, the poles of $\gamma$ must match, so either $\chi_1^{-1} \otimes \chi_1'$ or $\chi_1^{-1} \otimes \chi_2'$ must be unramified and infact equal to identity. Then we only have one character and can use stability. If we end up with $\chi_1 = \chi_2$ we get two poles after twisting and same on the other side.
\item $\rho$ is a $2$ dimensional tamely ramified irreducible induced by a tamely ramified character of an unramified quadratic extension; then twisting by a level $1$ character gives $|\epsilon| \sim q^{\frac{1}{2}-s}$. In this case, both characters must be induced by a tamely ramified character of the same unramified quadratic extension, so we can deal with the character instead of the dim $2$ irreducible. There we just use stability to prove the characters are the same. There will need to be a discussion of the Langlands constant and induction of $\epsilon$ here.
\end{itemize}
\end{itemize}
\item Do we ever actually need the finite field local converse theorem? 
\item Can we use finite field local converse theorem instead of stability?
\item Why does this proof technique break down eventually?
\item How far can it get? Can it prove $n = 3,4,5?$ Is $n=6$ when you start needing dim $2$ twists?
\item Can you get highly ramified $n=2$? Stability doesn't work as well here because of the level restrictions. What changes with non-semisimple $n=2$?
\end{itemize}
\endinput