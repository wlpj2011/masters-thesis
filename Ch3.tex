\chapter{Local Converse Theorem for Smooth Representations of the Weil Group}	%Chapter title
Definition of Weil Group \\
Definition of Smooth Representation of Weil Group \\
Definition of Inertia subgroup \\
Definition of L function of representation \\
Definition of $\epsilon$ factor of representation \\
Definition of $\gamma$ factor of representation \\
Definition of $\epsilon_0$ factor of representation \\
Stability Thoerem for characters of $\mathbb{F}^\times$ from BH 23.8\\
1D Local Converse Theorem \\
2D Local Converse Theorem \\
\begin{itemize}
\item $\gamma$ has $2$ poles, 
\begin{itemize} 
\item $\rho$ is the sum of two unramified characters
\end{itemize}
\item $\gamma$ has $1$ pole
\begin{itemize}
\item $\rho$ is the sum of an unramified character and a tamely ramified character
\item $\rho$ is the sum of two unramified characters, one of which is $1/q$ times the other cancelling out a pole and a zero in $\gamma$
\end{itemize}
\item $\gamma$ has no poles:
\begin{itemize}
\item $\rho$ is the sum of two tamely ramified characters; then twisting by a level $1$ character gives $|\epsilon| \sim q^{1-2s}$
\item $\rho$ is a $2$ dimensional tamely ramified irreducible induced by a tamely ramified character of an unramified quadratic extension; then twisting by a level $1$ character gives $|\epsilon| \sim q^{\frac{1}{2}-s}$
\end{itemize}
\end{itemize}
\endinput