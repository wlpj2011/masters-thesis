\chapter{Local Converse Theorem for Smooth Representations of the Weil Group}	%Chapter title

Most of the introductory material on local fields will echo that in Jean-Pierre Serre's \textit{Local Fields} \cite{Serre1979}, with the material on local class field theory coming from from Bushnell and Henniart's \textit{The Local Langlands Conjecture for $GL(2)$} Chapter 29 \cite{Bushnell2006}.
Adopting the notation in the second of these references, we will have the following notation for local fields.\\
$F$ and $E$ will refer to a non-Archimedean local fields.\\
$\fo$ is the discrete valuation ring in $F$.\\
$\fp$ is the unique maximal ideal of $\fo$.\\
$\varpi$ is the uniformizer and a generator of $\fp$.\\
$v_F$ will be that valuation defined on $F$ by $\fo$.\\
$\bk = \fo/\fp$ is the residue field of $F$.\\
$q = p^n = |k|$ is the size of the residue field where $p$ is the characteristic.\\
$U_F$ is the group of units of $\fo$.\\
$U_F^n = 1 + \fp^n$ for $n \geq 1$ are subgroups of the unit group forming a filtration.\\
$\psi$ will be an additive character of $F$ and $\widehat{F}$ will be the group of additive characters of $F$. \\
$\chi$ and $\xi$ will be characters of $F^\times$ or of $\cW_F$, frequently viewed as equivalent through Local Class Field Theory, and $\widehat{F^\times}$ will be the group of multiplicative characters of $F$. \\
$1_F$ will be the trivial character on $F^\times$ or $\cW_F$, which takes the value $1$ everywhere.
$\Omega_F$ will be the absolute Galois group of $F$ and $\cW_F$ will be the Weil group of $F$. \\
$\rho$ and $\sigma$ will be semisimple representations of $\cW_F$ and $\rho$ will be tamely ramified.\\


\section{Local Fields}
\label{sec:local-fields}
We are interested in a local converse theorem for smooth, tamely ramified, semisimple, $2$-dimensional representations of the Weil group of a local field.
First, we need to be able to define all the terms in stating the theorem.
Even though we are proving a theorem about representations of the Weil group $\cW_F$ of a local field $F$, the terms in the theorem must first be defined for the mulitplicative group $F^\times$ of the field.
\begin{defn}
  A \textbf{discrete valuation ring} is a principal ideal domain $\fo$ with a unique non-zero prime ideal $\fp$.
\end{defn}
We also call such a ring $\fo$ a DVR.
An alternative characterization of a DVR is that it is a domain $\fo$ such that its field of fractions $K$ has a non-trivial valuation $v : K \to \bbZ \cup \{ \infty \}$; that is, a function $v$ such that $v(xy) = v(x) + v(y)$, $v(x+y) \geq \min \{v(x),v(y)\}$, $v(x) = 0$ if and only if $x = 0$, and $v$ takes more values than just $0$ and $\infty$.
\begin{defn}
  A (non-Archimedean)\textbf{local field} is a field $F$ with a valuation $v$ that is locally compact with respect to the topology provided by the valuation and has finite residue field $k$ with $|k| = q = p^n$.
  The valuation defines a basis of open sets as the additive cosets of the $v^{-1}(\{x \in F \mid q^{-v(x)} \leq r\})$ for positive real numbers $r$.
\end{defn}
Given a local field $F$, it is not difficult to show that $\fo = \{x \in F \mid q^{-v(x)} \leq 1\}$ is the ring of integers of $F$ and is a DVR.
Further, $\fp = \{x \in F \mid q^{-v(x)} < 1\}$ is unique maximal (prime) ideal of the DVR $\fo$.
Finally, $\fo^\times = U_F = \{x \in F \mid q^{-v(x)} = 1\} = \{x \in F \mid v(x) = 0\}$ is the unit group of $\fo$.
A local field is isomorphic as a topological field to a finite extension of the $p$-adic numbers $\bbQ_p$ or the field of formal Laurent series $\bbF_q((T))$ over a finite field.
\\

For a DVR $\fo$ or for a local field $F$, we define a special element called the \textbf{uniformizer}, denoted by $\varpi$, which is a prime in $\fo$, so that $\varpi \fo = \fp$ is the unique prime ideal.
This choise is unique up to units.

\section{$L$ Functions and $\epsilon$ Factors of Local Fields}
\label{sec:Leps-LF}
Though we are interested in the representation theory of the Weil group, and how we can tell apart the representations of the Weil group, in order to define the objects we will use to do so, we must first define them for local fields.
Generally, a local converse theorem for local fields would deal with representations of $GL_n(F)$; however, all of the tools we need are developed while studying $GL_1(F) = F^\times$.\\

An additive character of a local field $F$ is a continuous homomorphism $\psi: F \to \bbC^\times$; these are the $1$-dimensional representations of the additive group of $F$.
Equivalently to saying $\psi$ is continuous, we can say $\ker \psi$ is open in the topology defined by the valuation.
We let $\hat{F}$ be the group of additive characters of $F$, which are a group under pointwise multiplication.
\begin{defn}
  Let $\psi \in \hat{F}$ with $\psi \neq 1$.
  The \textbf{level} of $\psi$ is the smallest integer $d$ such that $\fp^d \subseteq \ker \psi$.
\end{defn}
We then have the following proposition mostly characterizing the additive characters of $F$.
\begin{prop}
  Let $\psi \in \hat{F}$ with $\psi \neq 1$ be a level $d$ character.
  \begin{enumerate}
    \item Let $a \in F$. The map $a \psi : x \mapsto \psi(ax)$ is a character of $F$. If $a \neq 0$, then the character $a \psi$ has level $d - v_F(a)$, where $v_F$ is the valuation on $F$.
    \item The map $a \mapsto a \psi$ is a group isomorphism $F \cong \hat{F}$.
  \end{enumerate}
\end{prop}

We also define multiplicative characters of $F$ which are the $1$-dimensional representations of the multiplicative group of $F$.
In this setting, a multiplicative character of a local field $F$ is a continuous homomorphism $\chi : F^\times \to \bbC^\times$.
We also define level for multiplicative characters with a slight modification.
\begin{defn}
  Let $\chi$ be a non-trivial character of $F^\times$.
  The \textbf{level} of $\chi$ is defined to be the smallest integer $n \geq 0$ such that $U^{n+1}_F \subseteq \ker \chi$.
  We further say that $\chi$ is \textbf{unramified} if $U_F \subseteq \ker \chi$.
\end{defn}

The final thing we need to define the functions of interest is the concept of duality for representations of $F^\times$.
Because we are working with topological spaces, we need a definition of duality that respects the topological structure which will be called the \textbf{smooth dual} of a character $\chi$ and will be denoted by $\check{\chi}$.
If $(\chi, V)$ is a smooth representation of $F^\times$, then let $V^* = \hom_\bbC(V,\bbC)$ and let $(v^* , v) \mapsto \langle v^*, v\rangle$ be the canonical evaluation map.
Then we can define a representation $\chi^*$ of $F^\times$ on the space $V^*$ by
\[\langle \chi^*(g)v^*, v\rangle = \langle v^* , \chi(g^{-1})v\rangle.\]
This is not necessarily a smooth representation, but we can define $\check{V} = \cup_K (V^*)^K$ where $K$ ranges over compact open subgroups of $F^\times$ and the $(V^*)^K$ are the subspaces of $K$ fixed vectors in the representation.
Then we define $\check{\chi}$ as the restriction to $\aut_\bbC(\check{V})$.
This $(\check{\chi},\check{V})$ is the smooth dual of the character $(\chi, V)$.
\\

Now we can define the $L$-function and the  $\epsilon$-factor and $\gamma$-factor for characters of $F^\times$.
We define the $L$-function for a character of $F^\times$ as a variable of a complex variable $s$.

\begin{defn}
  Let $\chi$ be a characters of $F^\times$ and $q$ be the size of the residue field of $F$. Then we define the \textbf{$L$-function of $\chi$} as $L(\chi, s) : \bbC \to \bbC$ by 
  \[L(\chi, s) = \begin{cases} (1 - \chi(\varpi) q^{-s})^{-1} & \text{ if } \chi \text{ is unramified,} \\ 1 & \text{ otherwise} \end{cases}.\]
  This will be independent of the choice of uniformizer because unramified characters $\chi$ are trivial on units.
\end{defn}

Next, we define the $\epsilon$-factor as another function of a complex variable which is a sum of over the units of the ring of integers of $F$.
\begin{defn}
  For characters $\chi$ of $F^\times$ with level $n \geq 0$ and not unramified, and $\psi \in \hat{F}$ with level one, we define the \textbf{$\epsilon$-factor of $\chi$ (relative to $\psi$)} as
  \[\epsilon(\chi, s, \psi) = q^{n(\frac{1}{2} - s)} \sum_{x \in U_F/U_F^{n+1}} \chi(\alpha x)^{-1} \psi(\alpha x)/q^{(n+1)/2}\]
  for any $\alpha \in F^\times$ such that $v(\alpha) = -n$.
  For characters $\chi$ of $F^\times$ that are unramified; we define the $\epsilon$-factor by
  \[\epsilon(\chi, s, \psi) = q^{s - \frac{1}{2}} \chi(\varpi)^{-1}.\]
\end{defn}

As a quick justifcation for only dealing with $\psi$ of level one, we have the following lemma which describes how $\epsilon(\chi, s, \psi)$ changes as $\psi$ changes.

\begin{lemma}\label{lem:level-one-psi-reduction}
  Let $a \in F^\times$; then for fixed $\chi$, we have
  \[\epsilon ( \chi, s, a \psi) = \chi(a) \| a \|^{s - \frac{1}{2}} \epsilon (\chi, s, \psi)\]
\end{lemma}

Finally, we will define the $\gamma$ factor as a useful combination of the $L$-functions and $\epsilon$-factors.
\begin{defn}
  For characters $\chi$ of $F^\times$ and $\psi$ of $F$, then we define the \textbf{$\gamma$-factor of $\chi$ (relative to $\psi$)} by
  \[\gamma(\chi, s, \psi) = \epsilon(\chi, s, \psi) \frac{L(\check{\chi}, 1-s)}{L(\chi,s)}.\]
\end{defn}
The $\gamma$-factor encodes information about how $\chi$ acts on the uniformizer $\varpi$ and how $\chi$ acts on the units $U_F$ in such a way that we hope the $\gamma$-factor can help distinguish characters of $F^\times$.


As well as the definitions of $L, \gamma$, and $\epsilon$, we will need a theorem that describes how $\epsilon$ factors change as we twist them by high level characters; this is the stability theorem.

\begin{theorem}[Stability Theorem]
  \label{thm:Stability}
  Let $\theta, \chi$ be characters of $F^\times$ of level $ l \geq 0$ and $n \geq 1$.
  Suppose that $2l < n$.
  Let $\psi \in \hat{F}$ with $\psi \neq 1$ and let $c \in F$ satisfy $\chi(1+x) = \psi(cx)$ for $x \in \fp^{\lfloor n/2 \rfloor + 1}$.
  Then
  \[ \epsilon(\theta \chi, s, \psi) = \theta(c)^{-1} \epsilon(\chi, s, \psi).\]
\end{theorem}

To prove this; we first convert the $\epsilon$-factor into a Gauss sum and then prove a lemma about this Gauss sum.
If we define the \textbf{Gauss sum of $\chi$ (relative to $\psi$)} as
\[\tau(\chi, \psi) = \sum_{x \in U_F/U_F^{n+1}} \check{\chi}(cx) \psi(cx),\]
then this allows us to take the $s$ dependence out of the $\epsilon$-factor.
Specifically, we have from our definition above that for ramified $\chi$ of level $n \geq 0$
\[\epsilon(\chi, s, \psi) = q^{n(\frac{1}{2} - s)} \tau(\chi, \psi)/ q^{(n+1)/2}.\]
Then we have the following lemma

\begin{lemma}\label{lem:Gauss-sum-reduction}
  Suppose that $\chi$ has level $n \geq 1$.
  Let $c \in F$ satisfy
  \[ \chi(1 +x) = \psi(cx), \textrm{for all } x\in \fp^{\lfloor n/2 \rfloor + 1}\]
  Then
  \[\tau(\chi, \psi) = q^{\lfloor (n+1)/2 \rfloor} \sum_y \check{\chi}(cy) \psi(cy)\]
  where $y \in U_F^{\lfloor (n+1)/2 \rfloor} / U_F^{\lfloor n/2 \rfloor + 1}$.
\end{lemma}

\begin{proof}
  Recall that for a level $n$ character $\chi$
  \[\tau(\chi, \psi) = \sum_{x \in U_F/U_F^{n+1}} \check{\chi}(cx) \psi(cx).\]
  We will make the change of variable $x = y(1+z)$ with $y \in U_F/U_F^{\lfloor n/2 \rfloor + 1}$ and $z \in \fp^{\lfloor n/2 \rfloor + 1}/\fp^{n+1}$. Then we can rewrite the $\check{\chi}(cx) \psi(cx)$ in the sum as
  \[\check{\chi}(cy(1+z))\psi(cy(1+z)) = \check{\chi}(cy)\psi(cy) \psi(c(y-1)z).\]
  This follows because $\check{\chi}(cy(1+z)) = \check{\chi}(cy) \check{\chi}(1+z) = \check{\chi}(cy)\chi(1+z)^{-1}$ and by the assumptions of the lemma, $\chi(1+z)^{-1} = \psi(cz)^{-1} = \psi(-cz)$.
  Now, we can rewrite the sum as
  \[\tau(\chi, \psi) = \sum_x \check{\chi}(cx) \psi(cx) = \sum_y \check{\chi}(cy) \psi(cy)\left( \sum_z \psi(c(y-1)z)\right).\]
  The sum over $y \in U_F/U_F^{\lfloor n/2 \rfloor +1}$ looks superficially like what we want; but note that the desired sum is actually over $y \in U_F^{\lfloor (n+1)/2 \rfloor} /U_F^{\lfloor n/2 \rfloor +1}$.
  So we need to show that the sum over $z \in \fp^{\lfloor n/2 \rfloor +1} / \fp^{n+1}$ works out as necessary.
  Since $c(y-1) \in \fp^{-n}$, we find that $z \mapsto \psi(c(y-1)z)$ is a character of $\fp^{\lfloor n/2 \rfloor +1}/\fp^{n+1}$.
  In fact, $z \mapsto \psi(c(y-1)z)$ will be the trivial character if and only if $y \equiv 1 \mod \fp^{n - \lfloor n/2 \rfloor}$; i.e. $y \in U_F^{\lfloor (n+1)/2\rfloor}$.
  Since the sum of a character over its group is $0$ except for the trivial character; we find that this sum vanishes except when $y \in U_F^{\lfloor (n+1)/2 \rfloor}$ when the sum is $(\fp^{\lfloor n/2 \rfloor +1} : \fp^{n+1}) = q^{\lfloor (n+1)/2 \rfloor}$.
  Making this substitution, we find as desired that
  \[ \tau(\chi, \psi) = q^{\lfloor (n+1)/2\rfloor} \sum_y \check{\chi}(cy) \psi(cy) \]
  where the sum is now over $y \in U_F^{\lfloor (n+1)/2 \rfloor} /U_F^{\lfloor n/2 \rfloor +1}$.
\end{proof}

With this lemma in hand, we can now provide a proof of the stability theorem.

\begin{proof}
  We only need to consider $\psi$ level one because Lemma \ref{lem:level-one-psi-reduction} tells us how to convert to $\psi$ of other levels.
  Since $\theta$ is level $l$ and $\chi$ is level $n$ with $2l < n$, we have that $\theta\chi$ will be a level $n$ character because it is trivial on $U_F^{n+1} \subset U_F^{l+1}$.
  Similarly, we have that $\theta \chi$ agrees with $\theta$ on $U_F^{l+1}$.
  More usefully, because $2l < n$, we have that $\theta \chi$ agrees with $\theta$ on $U_F^{\lfloor (n+1)/2 \rfloor}$.
  Applying Lemma \ref{lem:Gauss-sum-reduction} to the Gauss sum portion of $\epsilon(\theta\chi,s,\psi)$, we get
  \[\tau(\theta \chi, \psi) = q^{\lfloor (n+1)/2\rfloor} \sum_y \check{\theta}\check{\psi}\]
  with the sum over $y \in U_F^{\lfloor (n+1)/2 \rfloor}/U_F^{\lfloor n/2 \rfloor + 1}$.
  Since $\theta$ is trivial on $U_F^{\lfloor (n+1)/2 \rfloor}$, this becomes $\theta(c)^{-1} \tau(\chi, \psi)$ as we expect.
  Substituting this into the expression for the $\epsilon$-factor, we find that
  \[\epsilon(\theta\chi, s, \psi) = \theta(c)^{-1} \epsilon(\chi, s, \psi)\]
  as desired.
\end{proof}



\section{Weil Groups of Local Fields}
\label{sec:weil-group}
Let $F$ be a non-Archimedean local field and pick a separable algebraic closure $\overline{F}$ of $F$.
Then we can define the absolute Galois group $\Omega_F = \gal(\overline{F}/F)$ which gets a natural topology as
\[\Omega_F = \lim_{\leftarrow} \gal(E/F)\]
where $E/F$ ranges over finite Galois extensions with $E \subseteq \overline{F}$.

For each $m \geq 1$, $F$ will have a unique unramified extensions $F_m/F$ of degree $m$ with $F_m \subseteq \overline{F}$.
Let $F_\infty$ be the composite of all these fields.
Then $F_\infty/F$ will be the unique maximal unramified extensions of $F$ in $\overline{F}$.
Each of the unique subextensions $F_m/F$ has a cyclic Galois group $\gal(F_m/F)$.
Each automorphism in $\gal(F_m/F)$ is determined by its action on the residue field $\bk_{F_m}$, which is isomorphic to $\bbF_{q^m}$ because $F_m/F$ is an unramified extension.
So, there will be a unique element $\phi_m \in \gal(F_m/F)$ that acts on $\bk_{F_m}$ by $x \mapsto x^q$.
We then let $\Phi_m = \phi_m^{-1}$.
We get a canonical isomorphism $\gal(F_m/F) \to \bbZ/m\bbZ$ by the map $\Phi_m \mapsto 1$; and by taking the limit over $m$; we get an isomorphism
\[\gal(F_\infty/F) \cong \lim_{\leftarrow m \geq 1} \bbZ/m\bbZ \cong \hat{\bbZ}\]
and a unique $\Phi_F \in \gal(F_\infty/F)$ that acts like $\Phi_m$ on each $F_m$.
We call this $\Phi_F$ the \textbf{geometric Frobenius substitution} on $F_\infty$.
Similarly, we define $\phi_F = \Phi_F^{-1}$ to be the \textbf{arithmetic Frobenius substitution} on $F_\infty$.
If an element of $\Omega_F$ has $\Phi_F$ as its image in $\gal(F_\infty/F)$, we call it a \textbf{geometric Frobenius element}.
We then define $\cI_F = \gal(\overline{F}/F_\infty)$ to be the \textbf{inertia group} of $F$.
This subgroup $\cI_F \subseteq \Omega_F$ roughly corresponds to the units $U_F$ of $F$.
There is another subgroup called the wild inertia group, denoted $\cP_F$, which roughly corresponds to the units $U_F^1$, and is the unique pro $p$-Sylow subgroup of $\cI_F$.
These correspondences are made precise in Theorem \ref{thm:LCFT}
\\

We can then define the Weil group $\cW_F$ as a subgroup of $\Omega_F$.
We first define ${}_a\cW_F$ as the inverse image in $\Omega_F$ of the subgroup $\langle \Phi_F\rangle$ of $\gal(F_\infty/F)$.
This is a dense normal subgroup of $\Omega_F$ generated by the Frobenius elements.
We then define the \textbf{Weil group of F} as a topological group with ${}_a\cW_F$ as the underlying abstract group, where $\cI_F$ is an open subgroup of $\cW_F$ and the topology on $\cI_F$ as a subspace coincides with the natural topology on $\cI_F$ as a subspace of $\Omega_F$.
\\

We then have a proposition defining a few properties of the Weil group relating to field extensions $E/F$.
\begin{prop}
  \begin{enumerate}
    \item Let $E/F$ be a finite extension with $E \subseteq \overline{F}$.
    \begin{enumerate}
      \item The group $\cW_F$ has a unique subgroup $\cW^E_F$ such that \[\bm{\iota}_F(\cW^E_F) = {}_a\cW_F \cap \Omega_E\] where $\bm{\iota}$ is the identity map $\cW_F \to {}_a\cW_F \subseteq \Omega_F$.
      \item The subgroup $\cW^E_F$ is open and of finite index in $\cW_F$; it is normal in $\cW_F$ if and only if $E/F$ is Galois.
      \item The canonical map $\cW^E_F\setminus \cW_F \to \Omega_E \setminus \Omega_F$ is a bijection.
      \item The canonical map $\bm{\iota}_E : \cW_E \to \Omega_E$ induces a topological isomorphism $\cW_E \cong \cW^E_F$.
    \end{enumerate}
    \item The map $E/F \mapsto \cW^E_F$ is a bijection between the set of finite extensions $E$ of $F$ inside $\overline{F}$ and the set of open subgroups of $\cW_F$ of finite index.
  \end{enumerate}
\end{prop}

Due to this proposition, we indentify $\cW_E$ with the subgroup $\cW^E_F$ of $\cW_F$ going forward.



The representations of $\cW_F$ form a particularly nice subcollection of the representations of $\Omega_F$, and it is these representations that we will be studying.
\section{Local Class Field Theory}
\label{sec:LCFT}
\begin{theorem}
\label{thm:LCFT}
  There is a canonical continuous group homomorphism
  \[\bm{a}_F : \cW_F \to F^\times\]
  with the following properties.
  \begin{enumerate}
    \item The map $\bm{a}_F$ induces a topological isomorphism $\cW_F^{\text{ab}} \cong F^\times$.
    \item An element $x \in \cW_F$ is a geometric Frobenius if and only if $\bm{a}_F(x)$ is a prime element of $F$.
    \item We have $\bm{a}_F(\cI_F) = U_F$ and $\bm{a}_F(\cP_F) = U^1_F$.
    \item If $E/F$ is a finite separable extensions, the diagram \[\begin{tikzcd}[ampersand replacement=\&,cramped]
	{\mathcal{W}_E} \&\& {E^\times} \\
	\\
	{\mathcal{W}_F} \&\& {F^\times}
	\arrow["{\bm{a}_E}", from=1-1, to=1-3]
	\arrow["{\bm{a}_F}"', from=3-1, to=3-3]
	\arrow[from=1-1, to=3-1]
	\arrow["{N_{E/F}}", from=1-3, to=3-3]
\end{tikzcd}\] commutes.
    \item Let $\alpha : F \to F'$ be an isomorphism of fields. The map $\alpha$ induces an isomorphism $\alpha: \cW^{\text{ab}}_F \to \cW^{\text{ab}}_{F'}$, and the diagram \[\begin{tikzcd}[ampersand replacement=\&,cramped]
	{\mathcal{W}_F^{\textrm{ab}}} \&\& {\mathcal{W}_{F'}^{\textrm{ab}}} \\
	\\
	{F^\times} \&\& {F'^\times}
	\arrow["\alpha", from=1-1, to=1-3]
	\arrow["\alpha"', from=3-1, to=3-3]
	\arrow["{\bm{a}_F}"', from=1-1, to=3-1]
	\arrow["{\bm{a}_{F'}}", from=1-3, to=3-3]
\end{tikzcd}\]
      commutes.
  \end{enumerate}
\end{theorem}

One consequence of Theorem \ref{thm:LCFT} is that the map $\bm{a}_F$, which we call the Artin reciprocity map, gives an isomorphism $\chi \mapsto \chi \circ \bm{a}_F$ between the group of characters of $F^\times$ and the group of characters of $\cW_F$.
More precisely, we have that unramified characters of $F^\times$ (trivial on $U_F$) correspond to unramified characters of $\cW_F$ (trivial on $\cI_F$); and tamely ramified characters of $F^\times$ (trivial on $U_F^1$, level $n=0$) correspond to characters of $\cW_F$ trivial on $\cP_F$.
We can use these correspondences to define the $L$-functions, $\epsilon$-factors, and $\gamma$-factors for representations of the Weil group.

\section{$L$-Functions and $\epsilon$-Factors of Weil Groups}
\label{sec:Leps-weil-group}
\begin{defn}
  If $\chi$ is a character of $\cW_F$, then we define
  \begin{align*}
    L(\chi, s) &= L(\chi \circ \bm{a}_F, s) \\
    \epsilon(\chi, s, \psi) &= \epsilon(\chi \circ \bm{a}_F, s, \psi)
  \end{align*}
  Where the functions on the right are those defined in Section \ref{sec:Leps-LF}.
\end{defn}
Going forward, we will use $\chi$ instead of $\chi \circ \bm{a}_F$ when it is clear we mean the character of $F^\times$ corresponding to the character of $\cW_F$.
Now that we have defined $L$ and $\epsilon$ for $1$ dimensional characters, we can extend their definitions to $n$-dimensional representations of $\cW_F$.

The $L$ function is then easy to extend to semisimple representations of $\cW_F$.
We simply say that $L(\sigma,s) = 1$ for irreducible representations $\sigma$ with dimensions $\geq 2$.
Then we make $L$ multiplicative by requiring that
\[L(\sigma_1 \oplus \sigma_2,s) = L(\sigma_1,s)L(\sigma_2,s).\]
\\

It takes more work to define the $\epsilon$-factor for all semisimple representations of $\cW_F$.
The main properties we need are that if $\rho_1, \rho_2$ are semisimple representations of $\cW_f$,then 
\[\epsilon(\rho_1 \oplus \rho_2,s,\psi_F) = \epsilon(\rho_1,s,\psi_F)\epsilon(\rho_2,s,\psi_F),\]
which is the multipicativity that we expect.
We also have an additional property that allows for induction of the local constant $\epsilon$ for characters.
This induction is the following property:

\begin{prop}\label{prop:local-constant-induction}
  If $\rho$ is a semisimple $n$-dimensional representation of $\cW_E$ and $E\supset F$, then
  \[\frac{\epsilon(\ind_{E/F} \rho, s, \psi_F)}{\epsilon(\rho,s,\psi_E)} = \frac{\epsilon(R_{E/F},s,\psi_F)^n}{\epsilon(1_E,s,\psi_E)^n}.\]
  Where $1_E$ is the trivial character on $\cW_E$ and $R_{E/F} = \ind^E_F 1_E$.
\end{prop}
Multiplicativity and Proposition \ref{prop:local-constant-induction} allows us to define the $\epsilon$-factor for any semi-simple representation because any irreducible representation of $\cW_F$ is the induced representation of a character from an appropriate finite extension.
Finally, as was the case for $F^\times$, we still define the $\gamma$-factor for semi-simple representation $\rho$ of $\cW_F$ as
\[\gamma(\rho, s ,\psi) = \epsilon(\rho, s, \psi) \frac{L(\check{\rho},1-s)}{L(\rho,s)}.\]

We also will find useful a few results from \cite{Bushnell2006} on which representations are induced from a representation of an extension.
First, we define an admissible pair.
\begin{defn}
  Let $E/F$ be a tamely ramified quadratic extension and $\chi$ a character of $E^\times$, we call $(E/F, \chi)$ an \textbf{admissible pair} if
  \begin{enumerate}
  \item $\chi$ does not factor through the norm map $N_{E/F}: E^\times \to F^\times$ and,
    \item if $\chi|_{U_E^1}$ does factor through $N_{E/F}$, then $E/F$ is unramified.
  \end{enumerate}
\end{defn}

We will let $\bbP_2(F)$ be the set of isomorphism classes of admissible pairs. Further, we will let $\cG_2^0$ be the set of isomorphism classes of irreducible $2$-dimensional representations of $\cW_F$ and $\cG_2^{\textrm{nr}}$ be the set of isomorphism classes $\rho \in \cG_2^0$ such that there is a non-trivial unramified character $\chi$ of $\cW_F$ such taht $\chi \otimes \rho \cong \rho$.
With these definition, we have the following theorem, which will allow us to work only with characters for $\cW_E$ for appropriate extensions $E/F$.
\begin{theorem}
  If $(E/F, \xi)$ is an admissible pair, the representation $\ind_{E/F} \xi$ of $\cW_F$ is irreducible. The map $(E/F, \xi) \mapsto \ind_{E/F} \xi$ induces a bijection

  \begin{align*}
    \bbP_2(F) &\to \cG_2^0(F) \,\,  \text{ if } p \neq 2 \text{, or} \\
    \bbP_2(F) &\to \cG_2^{\textrm{nr}}(F) \,\, \text{ if } p = 2.
  \end{align*}
\end{theorem} 

Finally, we have another theorem from \textcolor{red}{Add citation to essentially tame BH paper}.
For $\sigma \in \cG_2^0(F)$, they let $t(\sigma)$ be the number of (isomorphism classes) of unramified characters $\chi$ of $\cW_F$ such that $\sigma \otimes \chi \cong \sigma$.
They say that $\sigma$ is \textbf{essentially tame} if $p$ does not divide $n / t(\sigma)$, or equivalently, that $\sigma_{\cP_F}$ is a sum of characters.
Then the map $\bbP_2(F) \to \cG_2^{\textrm{et}}(F)$ defined by $(E/F, \xi) \mapsto \ind_{E/F} \xi$ is a bijection for all primes $p$ and all $n \geq 1$.
Since all tamely ramified $\rho$ are essentially tame; this tells use that all tamely ramified irreducible $\rho$ are induced from a tamely ramified character of an unramified extension $E/F$.
\textcolor{red}{Check this.}

\section{Local Converse Theorem for $2$ Dimensional Representations of the Weil Group}
\label{sec:n=2-LCT-weil-group}
Finally, we have all the pieces necessary to state the local converse theorem for representations of the Weil Group.
What has been proven in the most general case using the Langlands correspondence is the following theorem:

\begin{theorem}[Local Converse Theorem for Weil Groups]
  Let $\rho_1$ and $\rho_2$ be $n$-dimensional semisimple representations of $\cW_F$, with $n \geq 2$, such that for all semi-simple representations $\sigma$ of $\cW_F$ with dimension $k \leq \lfloor \frac{n}{2} \rfloor$ we have
  \[\gamma(\rho_1 \otimes \sigma,s,\psi) = \gamma(\rho_2 \otimes \sigma,s,\psi).\]
  Then $\rho_1 \cong \rho_2$.
\end{theorem}


What we would like to prove here is a local converse theorem specifically for the case $n=2$, which can be stated slightly more simply as follows:

\begin{theorem}[Local Converse Theorem on $\cW_F$ with $n=2$]
  \label{thm:LCTn=2}
  Let $\rho_1$ and $\rho_2$ be $2$-dimensional tamely ramified semisimple representations of $\cW_F$, such that for all characters $\chi$ of $\cW_F$, we have
  \[\gamma(\rho_1 \otimes \chi,s,\psi) = \gamma(\rho_2 \otimes \chi,s,\psi);\]
  then $\rho_1 \cong \rho_2$.
\end{theorem}

Note that we also restrict to tamely ramified representations in this theorem, though this only matters when the $\rho_i$ are irreducible.
Overall, this theorem can be split into three cases depending on the number of poles of $\gamma$, then two of those cases can be further split into two more cases each depending on the nature of the representations involved.
We shall prove each of those cases first as lemmas, then we shall prove that those are the only cases and so the local converse theorem on $\cW_F$ holds for $n=2$.

For the most part, we are not interested in how $\gamma$ behaves as a function of $s \in \bbC$, so we instead write $\gamma(\rho, X, \psi) = \gamma(\rho,s,\psi)$ where $X = q^{-s}$.
Then we also have that $\gamma(\rho,1-s,\psi) = \gamma(\rho,\frac{1}{qX},\psi)$ and the same for $\epsilon$ and $L$.
This simplifies the discussion around the number of poles of $\gamma$, so we can say that $\gamma$ has either $2$, $1$, or $0$ poles without worrying about the periodicity of $q^{-s}$.
In this notation, we will have the following definitions for $\epsilon$, $L$, and $\gamma$ restated here for simplicity.

As expected, the $\gamma$-factor modification is easy and we just get that \[\gamma(\chi, X, \psi) = \epsilon(\chi, X, \psi) \frac{L(\check{\chi}, \frac{1}{qX})}{L(\chi, X)}.\]
$L$ is simple as well and we find that 
\[L(\chi, X) = \begin{cases} (1 - \chi(\varpi) X)^{-1} & \text{ if } \chi \text{ is unramified,} \\ 1 & \text{ otherwise.} \end{cases}\]
Finally, the $\epsilon$-factor takes a bit more effort to convert to an expression in terms of $X$, but we find that
\[\displaystyle\epsilon(\chi, X, \psi) = \begin{cases} \frac{X^n}{\sqrt{q}} \sum_{x \in U_F/U_F^{n+1}} \chi(\alpha x)^{-1} \psi(\alpha x) & \text{ for } \chi \text{ of level } n \text{ and } v_F(\alpha) = -n \\ \frac{1}{\sqrt{q}X\chi(\varpi)} & \text{ for } \chi \text{ unramified.}\end{cases}\]



With these results, we can now begin proving the individual cases of Theorem \ref{thm:LCTn=2}.

\begin{lemma}[$\gamma$ has $2$ poles]
  \label{lem:gamma-two-poles}
  Suppose $\rho_1$ and $\rho_2$ are $2$-dimensional tamely ramified semisimple representations of $\cW_F$, such that
  \[\gamma(\rho_1,X,\psi) = \gamma(\rho_2,X,\psi),\]
  and $\gamma(\rho_i,X,\psi)$ has $2$ poles.
  Then $\rho_1 \cong \rho_2$.
\end{lemma}

\begin{proof}
  Suppose that $\gamma(\rho_1,X,\psi) = \gamma(\rho_2,X,\psi)$ has two poles.
  Recall that
  \[\gamma(\rho_i,X,\psi) = \epsilon(\rho_i,X,\psi) \frac{L(\check{\rho_i},\frac{1}{qX})}{L(\rho_i,X)}.\]
  So $\gamma(\rho_i,X,\psi)$ has poles only when $\epsilon(\rho_i,X,\psi)$ has poles, when $L(\check{\rho_i},\frac{1}{qX})$ has poles, and when $L(\rho_i,X)$ has zeroes.
  However, $\epsilon(\rho_i,X,\psi)$ never has poles because it's only $X$ dependence comes from the $X^n$ factor for not unramified level $n$ characters, or the $\frac{1}{\sqrt{q}X}$ factor for unramified characters.
  Note that as an actual complex function, $X \neq 0$ because $X = q^{-s}$, so we discount this possibility for a pole.
  Similarly, $L(\rho_i,X)$ never has zeroes because it is defined multiplicatively as $1$ for irreducible with dimension $> 1$ and ramified characters and $(1 - \chi(\varpi)X)^{-1}$ for unramified characters.
  Specifically, we see that $L(\check{\rho_i},\frac{1}{qX})$ has one pole for each unramified character in $\check{\rho_i}$.
  Since $L(\check{\rho_i}, \frac{1}{qX})$ is the only source of poles in $\gamma(\rho_i,X,\psi)$, we must have that $L(\check{\rho_i},\frac{1}{qX})$ has two poles and therefore $\rho_i$ each contain two unramified characters.
  Since the $\rho_i$ are two dimensional characters of $\cW_F$, we therefore have that each $\rho_i$ is in fact a direct sum of two unramified characters of $\cW_F$.

  Write $\rho_i = \theta_i \oplus \theta_i'$, with $\theta_i, \theta_i'$ both unramified characters of $\cW_F$.
  Since $\gamma(\rho_1,X,\psi) = \gamma(\rho_2,X,\psi)$ have the same two poles, we must have that $L(\check{\rho_1},\frac{1}{qX})$ and $L(\check{\rho_2},\frac{1}{qX})$ have the same poles which is equivalent to $L(\rho_1,X)$ and $L(\rho_2,X)$ having the same poles.
  Because the $L$-function is multiplicative we have that
  \[L(\theta_1,X)L(\theta_1',X) = L(\theta_2,X)L(\theta_2',X).\]
  Similarly, because we know that all the characters involved are unramified, we find that by treating the $\theta_i,\theta_i'$ as characters of $F^\times$ we have
  \[(1 - \theta_1(\varpi)X)^{-1}(1 - \theta_1'(\varpi)X)^{-1} = (1 - \theta_2(\varpi)X)^{-1}(1 - \theta_2'(\varpi)X)^{-1}.\]
  The left side has poles at $X =  \frac{1}{\theta_1(\varpi)}, \frac{1}{\theta_1'(\varpi)}$ and the right side has poles at $X =  \frac{1}{\theta_2(\varpi)}, \frac{1}{\theta_2'(\varpi)}$.
  These must be the same, so without loss of generality, we can say $\theta_1(\varpi) = \theta_2(\varpi)$ and $\theta_1'(\varpi) = \theta_2'(\varpi)$.
  However, unramified characters of $\cW_F$ correspond to unramified characters of $F^\times$ which are trivial on $U_F$.
  Since every element of $x \in F^\times$ can be written as $u\varpi^m$ for some $u \in U_F$ and $m \in \bbZ$, we have that
  \[\theta(x) = \theta(u\varpi^m) = \theta(u) \theta(\varpi)^m = \theta(\varpi)^m.\]
  So every unramified character of $F^\times$ is fully determined by its value on $\varpi$, and similarly for characters of $\cW_F$.
  This tells us that $\theta_1 = \theta_2$ and $\theta_1' = \theta_2'$, so we clearly find that \[\rho_1 \cong \rho_2\] which is what we wanted to show.
\end{proof}

Next is the case when $\gamma(\rho_i, X,\psi)$ only has one pole, which can occur in two different ways. 

\begin{lemma}[$\gamma$ has $1$ poles]
  \label{lem:gamma-one-pole}
  Suppose $\rho_1$ and $\rho_2$ are $2$-dimensional tamely ramified semisimple representations of $\cW_F$, such that for all characters $\chi$ of $\cW_F$, we have
  \[\gamma(\rho_1 \otimes \chi,X,\psi) = \gamma(\rho_2 \otimes \chi,X,\psi),\]
  and $\gamma(\rho_i,X,\psi)$ has $1$ poles.
  Then $\rho_1 \cong \rho_2$.
\end{lemma}

\begin{proof}
  Suppose $\rho_1$ and $\rho_2$ are $2$-dimensional tamely ramified semisimple representations of $\cW_F$, such that for all characters $\chi$ of $\cW_F$, we have
  \[\gamma(\rho_1 \otimes \chi,X,\psi) = \gamma(\rho_2 \otimes \chi,X,\psi),\]
  and $\gamma(\rho_i,X,\psi)$ has $1$ poles.
  As was mentioned in the proof of Lemma \ref{lem:gamma-two-poles}, the only source of poles in $\gamma(\rho_i,X,\psi)$ is the poles of $L(\check{\rho_i},\frac{1}{qX},\psi)$.
  So in order to have only one pole we must either have that the $\rho_i$ are the sum of an unramified character and a ramified tamely ramified character; or the $\rho_i$ are a sum of two unramified characters and that somehow a pole of $L(\check{\rho_i},\frac{1}{qX})$ cancels out with a zero of $L(\rho_i,X)$.
  We will need to show that those two cases cannot coexist, and then prove the converse theorem in each case.
  We can separate these cases by looking at the zeroes of $\gamma(\rho_i,X,\psi)$ as well as the poles.
  First, suppose $\rho_i = \theta_i \oplus \xi_i$ with $\theta_i$ unramified and $\xi_i$ ramified tamely ramified.
  Then we find that $\gamma(\theta_i \oplus \xi_i,X,\psi)$ has a pole at
  \[X= \frac{1}{q\theta_i(\varpi)}\]
  coming from $L(\check{\theta}_i,\frac{1}{qX})$ and a zero at
  \[X = \frac{1}{\theta_i(\varpi)}\]
  coming from $L(\theta_i,X)$.
  On the other hand, suppose that $\rho_i = \theta_i \oplus \theta_i'$ with $\theta_i$ and $\theta_i'$ both unramified.
  Then as before, we have that (without cancellation of zeroes and poles)  $\gamma(\rho_i,X,\psi)$ has poles at
  \[X = \frac{1}{q\theta_i(\varpi)}, \frac{1}{q\theta_i'(\varpi)}\]
  and has zeroes at
  \[X = \frac{1}{\theta_i(\varpi)}, \frac{1}{\theta_i'(\varpi)}.\]


  In order to only have a single pole in this case, we must have that one of the zeroes cancels out one of the poles.
  Without loss of generality, we say that 
  \[\theta_i'(\varpi) = q\theta_i(\varpi).\]
  Since an unramified character $\theta_i'$ is fully defined by the value it takes on $\varpi$, we have that there is only one choice for $\theta_i'$ for any given $\theta$.
  The left over zeroes and poles are then required to be a pole coming from $L(\check{\theta_i'},\frac{1}{qX})$ at 
  \[X = \frac{1}{q^2 \theta_i(\varpi)}\]
  and a zero coming from $L(\theta_i,X)$ at
  \[X = \frac{1}{\theta_i(\varpi)}\]
  What is of note here is that in the case when $\rho_i = \theta_i \oplus \xi_i$ is the sum of an unramified character and a ramified tamely ramified character, then the ratio between the pole and the zero of $\gamma(\rho_i,X,\psi)$ is $q$.
  On the other hand, when $\rho_i = \theta_i \oplus \theta_i'$ is the sum of two unramified characters with poles and zeroes that cancel out, then the ratio between the pole and the zero of $\gamma(\rho_i,X,\psi)$ is $q^2$.
  So we can tell these cases apart by looking at the zeroes and poles of $\gamma(\rho_i,X,\psi)$.


  Now, let us show that both of these types of representations satisfy the converse theorem.
  We will start with the case where $\rho_i = \theta_i \oplus \theta_i'$ is the sum of two unramified characters with $\theta_i'(\varpi) = q \theta_i (\varpi)$.
  Since we have
  \[\gamma(\theta_1 \oplus \theta_1', X, \psi) = \gamma(\theta_2 \oplus \theta_2',X,\psi)\]
  we again must have that the poles are equal on each side.
  As mentioned above, $\gamma(\theta_i \oplus \theta_i' ,X, \psi)$ must have a pole at $X = \frac{1}{q^2 \theta(\varpi)}$ if $\theta_i$ and $\theta_i'$ are both unramified and their poles and zeroes cancel.
  But \[\frac{1}{q^2 \theta_1(\varpi)} = \frac{1}{q^2 \theta_2(\varpi)}\]
  clearly implies $\theta_1(\varpi) = \theta_2(\varpi)$ and so $\theta_1 = \theta_2$ and $\theta_1' = \theta_2'$ because they are unramified and so determined by their value at $\varpi$.
  As desired, this gives us $\rho_1 = \rho_2$ for the unramified case.

  Next, consider the case of $\rho_i = \theta_i \oplus \xi_i$ with $\theta_i$ unramified and $\xi_i$ ramified tamely ramified (level $n = 0$).
  We have that $L(\rho_i,X) = L(\theta_i \oplus \xi_i ,X) = L(\theta_i,X)L(\xi_i,X) = L(\theta_i,X)$ because ramified characters have trivial $L$ functions.
  So we can still identify that the poles of $\gamma(\rho_1,X,\psi)$ and $\gamma(\rho_2,X,\psi)$ are the same so $\theta_1(\varpi) = \theta_2(\varpi)$ which like the previous cases tells us that $\theta_1 = \theta_2$, so just call it $\theta$.
  
  The conditions of the converse theorem tell us that
  \[ \gamma((\theta \oplus \xi_1) \otimes \chi,X,\psi) = \gamma((\theta \oplus \xi_2) \otimes \chi,X,\psi)\]
  for all characters $\chi$ of $\cW_F$.
  However, because $(\theta \oplus \xi_1)\otimes \chi = (\theta \otimes \chi) \oplus (\xi_1 \otimes \chi)$ and $\gamma$ is multiplicative, we have that
  \[\gamma (\xi_1 \otimes \chi,X,\psi) = \gamma(\xi_2 \otimes \chi,X,\psi)\]
  Now, consider twisting by $\chi = \xi_1^{-1}$.
  Then we have that \[\gamma(\xi_1 \otimes \xi_1^{-1},X,\psi) = \gamma(\xi_2 \otimes \xi_1^{-1} , X, \psi).\]
  $\xi_1 \otimes \xi_1^{-1}$ is the trivial character $1_F$ and is clearly unramified.
  Since $\gamma(\xi_2 \otimes \xi_1^{-1}, X, \psi)$ must have the same poles and zeros as $\gamma(1_F, X, \psi)$, we find that $\xi_2 \otimes \xi_1^{-1}$ must also be unramified, and agree with $1_F$ on $\varpi$, making $1_F = \xi_2 \otimes \xi_1^{-1}$.
  This is equivalent to $\xi_1 = \xi_2$.
  Combined with $\theta_1 = \theta_2$, we find that $\rho_1 \cong \rho_2$ as desired.
\end{proof}

Now we turn our attention to the final case when $\gamma$ has no poles, which is when there are no unramified characters present.
In this case we have the following lemma.

\begin{lemma}[$\gamma$ has $0$ poles]
  \label{lem:gamma-zero-poles}
  Let $F$ be a local field with residue field $\bk_F = \bbF_q$ where $q = p^n$ and $p \neq 2$.
  Suppose $\rho_1$ and $\rho_2$ are $2$-dimensional tamely ramified semisimple representations of $\cW_F$, such that for all characters $\chi$ of $\cW_F$, we have
  \[\gamma(\rho_1 \otimes \chi,X,\psi) = \gamma(\rho_2 \otimes \chi,X,\psi),\]
  and $\gamma(\rho_i,X,\psi)$ has $0$ poles.
  Then $\rho_1 \cong \rho_2$.
\end{lemma}

\begin{proof}
  Let $F$ be a local field with a residue field not of characteristic $2$, and suppose $\rho_1$ and $\rho_2$ are $2$-dimensional tamely ramified semisimple representations of $\cW_F$, such that for all characters $\chi$ of $\cW_F$, we have
  \[\gamma(\rho_1 \otimes \chi,X,\psi) = \gamma(\rho_2 \otimes \chi,X,\psi),\]
  and $\gamma(\rho_i,X,\psi)$ has $0$ poles.
  We will need to show that there is no way for all the poles to cancel out if one of the $\rho_i$ has an unramified character as a subrepresentation.
  Then the two ways in which we can have no poles are either that the $\rho_i$ are the sum of two ramified tamely ramified characters or that the $\rho_i$ are two dimensional ramified tamely ramified irreducibles.
  Next we will need a way to distinguish between these two cases; then we will show that the converse theorem holds in each case.

  The proof of Lemma \ref{lem:gamma-one-pole} shows us why it is not possible for $\gamma$ to have no poles if there is an unramified portion of the $\rho_i$.
  Now, suppose first that $\rho_i = \xi_i \oplus \xi_i'$ with $\xi_i, \xi_i'$ both tamely ramified (level n=0).
  Consider twisting by $\chi = \xi_i^{-1}$, then we get $\rho_i \otimes \chi = 1_F \oplus (\xi_i^{-1} \otimes \xi_i')$.
  Note that $\gamma(\rho_i \otimes \chi,X,\psi)$ will have at least one pole in this case.

  On the other hand, if $\rho_i$ is a two dimensional ramified tamely ramified irreducible representation of $\cW_F$, then there is some unramified quadratic extension $E_i/F$ and ramified tamely ramified character $\xi_i$ of $\cW_{E_i}$ such that $\rho_i = \ind_{E_i/F} \xi_i$.
  Since there is a unique unramified quadratic extension of $F$ inside a given separable closure $\overline{F}$, we just call it $E$ and let $\rho_i = \ind_{E/F}$.
  Recall that by Proposition \ref{prop:local-constant-induction} we have
  \[\frac{\epsilon(\ind_{E/K} \rho, s, \psi_K)}{\epsilon(\rho,s,\psi_E)} = \frac{\epsilon(R_{E/K},s,\psi_K)^n}{\epsilon(1_E,s,\psi_E)^n}.\]
  Rearranging this and rewriting in terms of $X$ and the representations we are interested in, we have that
  \[\epsilon(\rho_i,X_F,\psi_F) = \epsilon(\ind_{E/F} \xi_i, X_F, \psi_F) = \epsilon(\xi_i, X_E, \psi_E) \frac{\epsilon(\ind_{E/F} 1_E,X_F,\psi_F)^2}{\epsilon(1_E,X_E,\psi_E)^2},\]
  where $X_F = q^{-s}$ and $X_E = (q^2)^{-s}$.
  Now, note that $(\ind_{E/F} \xi_i )\otimes \chi = \ind_{E/F} (\xi_i \otimes \chi \circ N_{E/F})$.
  Since $L(\rho_i) = 1$ for $\rho_i$ irreducible dimension $ > 1$, we have that $\gamma(\rho_i,X,\psi)$ has no poles as expected for this case.
  However, we further have that $\gamma(\rho_i \otimes \chi,X,\psi)$ will never have a pole because $\ind_{E/F} (\xi_i)$ is irreducible if and only if $\xi$ is different than its conjugate, i.e.  $\xi_i \neq \xi_i^{q}$.
  This is also equivalent to saying that $\xi_i$ doesn't factor through the norm.
  So twisting a $\xi_i$ that doesn't factor through the norm by $\chi \circ N_{E/F}$ will never produce something that does factor through the norm, meaning that $(\ind_{E/F} \xi_i )\otimes \chi$ is irreducible and so $\gamma(\rho_i \otimes \chi, s, \psi)$ will never have a pole.
  This allows us to tell apart $\rho_i = \xi_i \oplus \xi_i'$ and $\rho_i = \ind_{E/F} \xi_i$; the first case will have some twist $\chi$ that gives $\gamma(\rho_i,X,\psi)$ at least one pole and the second case will never have such a $\chi$.

  
  Now, let us deal with the $\rho_i = \xi_i \oplus \xi_i'$ case with $\xi_i, \xi_i'$ ramified tamely ramified.
  This case works out much like the case when $\gamma$ has one pole and $\rho$ is a sum of an unramified and a ramified tamely ramified character.
  In fact, consider the twist by $\chi = \xi_1^{-1}$.
  We get \[\gamma(1_F \oplus (\xi_1' \otimes \xi_1^{-1}),X,\psi) = \gamma((\xi_2 \otimes \xi_1^{-1})\oplus(\xi_2' \otimes \xi_1^{-1}),X,\psi)\]
  Because $1_F$ is unramified, we must have at least one pole, possibly two poles.
  That is either we are in the case of Lemma \ref{lem:gamma-one-pole} or Lemma \ref{lem:gamma-two-poles}; either way, we find that, without loss of generality, $\xi_1 = \xi_2$ and $\xi_1' = \xi_2'$ and so $\rho_1 = \rho_2$.
  
  Finally, let us deal with the case when $\rho_i = \ind_{E/F} \xi_i$ with $\xi_i$ a tamely ramified character of $\cW_E$.
  We will first use stability to find the value of $\xi_i$ on $\varpi$, then we will use the Gauss sum local converse to show that $\xi_1$ and $\xi_2$ agree sufficiently on $U_F$ to have $\xi_1 \cong \xi_2$.

  Note that the parts in the fraction for the induction constant depend only on the extension $E$ which is fixed, we in fact have that if $\epsilon(\rho_1,X,\psi_F) = \epsilon(\rho_2,X,\psi_F)$, then $\epsilon(\xi_1 ,X,\psi_E) = \epsilon(\xi_2, X,\psi_E)$.
  Now, applying the Stability Theorem, Theorem \ref{thm:Stability}, for $E$ with a twist by the level $1$ character $\chi$ of $E$ defined by $\chi(1+x) = \psi(\varpi^{-1} x)$, we find that
  \[\xi_1(\varpi^{-1})^{-1}\epsilon(\chi,X,\psi_E) = \xi_2(\varpi^{-1})^{-1} \epsilon(\chi,X,\psi_E).\]
  This tells us that $\xi_1(\varpi) = \xi_2(\varpi)$ and so we only need to understand how they act on units.
  Since the $\xi_i$ are tamely ramified, we already know that they are trivial on $U_E^1 = U_F^1$, we only need to study the character's restriction of $U_E/U_E^1 \cong \bbF_{q^2}$.
  Similarly, when we restrict tamely ramified characters $\chi$ of $\cW_F$ to the units, we can in fact study how they act on $U_F/U_F^1 \cong \bbF_q$.
  This puts us in the case of the Gauss sum converse theorem where we have characters of $\bbF_{q^2}$ being twisted by characters of $\bbF_q$; and Lemma \ref{lem:Gauss-sum-reduction} provides this reduction for us.
  Applying that reduction, and then the Gauss sum local converse theorem proves that $\xi_1 \cong \xi_2$ or $\xi_1 \cong \xi_2^q$ on $U_E$.
  However, $\ind_{E/F} \xi_1$ and $\ind_{E/F} \xi_1^q$ are isomorphic, so we do indeed have that $\rho_1 \cong \rho_2$ as desired.

  
\end{proof}

Now that we have the necessary lemmas, we can assemble them into a proof of Theorem \ref{thm:LCTn=2}.
\begin{proof}
  Suppose that $\rho_1$ and $\rho_2$ are $2$-dimensional tamely ramifed semi-simple representations of $\cW_F$, such that for all characters $\chi$ of $\cW_F$ we have
  \[\gamma(\rho_1 \otimes \chi,X,\psi) = \gamma(\rho_2 \otimes \chi, X, \psi)\]
  As mentioned before, the only place poles can come from is the factor of $L(\check{\rho_i} \otimes \check{\chi},\frac{1}{qX})$, so there can be at most two poles of $\gamma(\rho_i \otimes \chi, X, \psi)$.
  In the case when there are two poles, Lemma \ref{lem:gamma-two-poles} proves the converse theorem without using any twists.
  Alternatively, this can be viewed as twisiting only by the trivial character which could be thought of as a zero dimensional thing.
  In the case where this is one pole, Lemma \ref{lem:gamma-one-pole} proves the converse theorem using at most $1$ twist.
  Finally, in the case where there are zero poles, Lemma \ref{lem:gamma-zero-poles} proves the converse theorem using all of the twists by characters since we must use them to distinguish irreducibles.
  Since we can have at most two poles and we can't have fewer than zero; this covers all posibilities and concludes the proof of Theorem \ref{thm:LCTn=2}.
\end{proof}

\endinput
