%\copyrightpage[...]{...}		% Optional, comment out or delete if undesired

\begin{abstract}
% Abstract text here, either typed in directly or included using an `\input{}' command
  A local converse theorem is a theorem which states that if two representations $\chi_1$, $\chi_2$ have equal $\gamma$-factors for all twists by representations $\sigma$ coming from a certain class, then $\chi_1$ and $\chi_2$ are equivalent in some way.
  We provide a direct proof of a local converse theorem in two distinct settings.
  Previous proofs published in the literature for these settings were indirect proofs making use of various correspondences between representations of other groups.
  \\

  We first prove a Gauss sum local converse theorem for representations of $\bbF_{p^2}^\times$ twisted by representations of $\bbF_p^\times$ where equality means that $\chi_1$ and $\chi_2$ are in the same Frobenius orbit.
  Specifically, we prove that if $\chi_1, \chi_2$ are regular representations of $\bbF_{p^2}^\times$ such that $S(\chi_1 \otimes \sigma \circ N_{\bbF_{p^2}/\bbF_{p}}) = S(\chi_2 \otimes \sigma \circ N_{\bbF_{p^2}/\bbF_{p}})$ for all representations $\sigma$ of $\bbF_p^\times$, then $\chi_1 = \chi_2$ or $\chi_1 = \chi_2^p$.
  We then apply this theorem to tamely ramified, $2$-dimensional representations of the Weil group $\cW_F$ for a local field $F$ where we show that if $\rho_1, \rho_2$ are $2$-dimensional, tamely ramified representations of $\cW_F$ such that for all $1$-dimensional representations $\chi$ of $\cW_F$ it is true that $\gamma(\rho_1 \otimes \chi, s, \psi) = \gamma(\rho_2 \otimes \chi , s, \psi)$, then $\rho_1 \cong \rho_2$.
\end{abstract}

%\begin{layabstract}{...}	% Replace the ... with the list of keywords
% Lay abstract text here, either typed in directly or included using an `\input{}' command
%To add later if necessary
%\end{layabstract}

% The optional preface, dedication, and acknowledgements environments are included similar to the abstract environment

%\begin{preface}
% Preface text here
%What needs to go here?
%\end{preface} 

%\begin{dedication}
% Dedication text here
%To add later
%\end{dedication}

\begin{acknowledgements}
% Acknowledgments text here
  This thesis would not have been possible without the assistance of many people at the University of Maine and elsewhere.
  First, I would like to thank my advisor, Professor Gilbert Moss, without whom I would have never would have found a thesis topic, nor made as much progress as I was able to this year.
  I would also like to thank the rest of my thesis committee, Professors Jack Buttcane and Tyrone Crisp, who have provided useful feedback throughout the process of writing my thesis.

  Additionally, I would like to thank my friends and family.
  My parents Sarah and Paul Johnson for providing motivation to complete my thesis even while managing other responsibilites.
  My friends Ethan King and Hamilton Wan for helping me prepare for my presentations.
  And finally, I would like to thank my cats Lucifer and Cassandra for comforting me throughout the process and ensuring that I never spent too much time at the computer without them by my side.
\end{acknowledgements}

% Commands for the required lists
\tableofcontents
%\listoftables				% Include only if there are tables in the thesis
%\listoffigures				% Include only if there are figures in the thesis

% If you have other lists which need to be included they go here, possibly using the listof environment
%\begin{thesislist}{...}		% Replace the ... with name of the things being listed here
% Contents of list
%\end{thesislist}

% Sets the document spacing and pagestyle.
\mainmatter

\endinput
