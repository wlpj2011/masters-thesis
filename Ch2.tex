\chapter{Local Converse Theorem for Finite Fields}	%Chapter title
\begin{itemize}
\item Finite Fields 
\begin{itemize}
\item additive characters of $\hat{\bbF}_p$
\item multiplicative characters of $\mathbb{F}_p^\times$ and $\mathbb{F}_{p^n}^\times$
\item Trace $\mathbb{F}_{p^n}^\times \to \mathbb{F}_{p}^\times$
\item Regular characters
\end{itemize}
\item Gauss Sums 
\item Teichmuller Characters 
\item Stickelberger Theorem 
\begin{itemize}
\item A description of the prime ideals involved?
\item $p$-adic expansions and how they relate to Gauss sums on a first order of approximation, especially if Gross-Koblitz gets included
\end{itemize}
\item Gross-Koblitz Formula if necessary 
\begin{itemize}
\item If I include this, then I need $p$-adic gamma function
\end{itemize}
\item Restate the converse theorem in terms of $s(k)$ instead of $S(\omega^{-k})$.
\item Do I need a $1$D analog converse for characters of $\mathbb{F}_p^\times$ or $\mathbb{F}_q^\times$ 
\item $2$D analog converse for characters of $\mathbb{F}_{p^2}^\times$ twisted by characters of $\mathbb{F}_{p}^\times$ 
\item If necessary, converse for characters of $\mathbb{F}_{q^2}^\times$ twisted by characters of $\mathbb{F}_{q}^\times$; need to look into a proof of this.
\item If possible, converse for characters of $\mathbb{F}_{p^3}^\times$ twisted by characters of $\mathbb{F}_{p}^\times$ 
\end{itemize}

\section{Gauss Sums for Finite Fields}
\noindent We will use the following notation in this chapter; mostly following the notation choices in \cite{Lang1990}.\\
$\bbF_p$ will be the finite field with $p$ elements.\\
$\bbF_q$ will be the finite field with $q = p^n$ elements for some $n \geq 1$.\\
For any finite field $\bbF_q$, $\bbF_q^\times$ will be the multiplicative unit group of the field.\\
$\varepsilon$ will be a primitive $p^\text{th}$ root of unity. For $\varepsilon \in \bbC$, we use $\varepsilon = e^{2 \pi i /p}$.\\
$\mu_N$ will be the group of $N^\text{th}$ roots of unity.\\

We will be dealing only with complex representations of finite fields.
Since $\bbF_q$ is a field, all representations are $1$ dimensional characters.
An additive character of $\bbF_q$ is a function $\xi : \bbF_q \to \bbC^\times$ with $\xi(a + b) = \xi(a)\xi(b)$.
A multiplicative character of $\bbF$ is a function $\chi : \bbF_q^\times \to \bbC^\times$ with $\chi(ab) = \chi(a)\chi(b)$.
If we ever need to evaluate a multiplicative character $\chi$ at $0 \not \in \bbF_q^\times$ then we use $\chi(0) = 0$. \\

The two most important characters in this setting are the additive character defined by the trace and the multiplicative character called the Teichm\"uller character. 
The primary trace we are interested in is the trace for the extension $\bbF_q / \bbF_p$ which is defined in the following way.
%rephrase this to remove defn environment; only want defn env when will refer to it later
\begin{defn}
The trace from $\bbF_q / \bbF_p$ for $q = p^n$ is defined by $\tr : \bbF_q \to \bbF_p$
 \[ \tr : x \mapsto \sum_{i=0}^{n-1} x^{p^i} = x + x^p + x^{p^2} + \cdots + x^{p^{n-1}}.\]
\end{defn}
We can also define a trace for any extension; specifically, for the extension $\bbF_{q^n}/\bbF_q$ we have a trace defined by \[\tr_{\bbF_{q^n}/\bbF_q} : x \mapsto \sum_{i=0}^{n-1} x^{q^i}.\] 
We can unify these definitions of trace in the following way.
\begin{defn}
Let $E/F$ be a Galois field extension and let $G = \gal(E/F)$ be the Galois group of automorphisms of $E$ fixing $F$. 
Then the trace $\tr : E \to F$ is defined by \[\tr : x \mapsto \sum_{\sigma \in G} \sigma x.\]
\end{defn}

\noindent Then we can define an additive character $\lambda : \bbF_q \to \mu_p$ by $\lambda(x) = \varepsilon^{\tr(x)}$.
It takes a bit more to define the Teichm\"uller character.\\

Consider the field $\bbQ(\mu_{q-1})$ and let $\fp$ be the prime ideal lying over $p$.
Then the residue field of $\bbQ(\mu_{q-1})$ mod $\fp$ is isomorphic to $\mu_{q-1}$ which in turn is isomorphic to $\bbF_q^\times$.
\begin{defn}\label{def:Teichmuller-char}
We define the Teichm\"uller character $\omega : \bbF_q^\times \to \mu_{q-1}$ by \[\omega (u) \equiv  u \mod \fp.\]
\end{defn}
\noindent This character generates the character group of $\bbF_q^\times$; so for all multiplicative characters $\chi$ of $\bbF_q^\times$ there is some integer $k$ such that $\chi = \omega^k$.\\
When we have two multiplicative character $\chi_1$ and $\chi_2$, we say that the twist of $\chi_1$ by $\chi_2$ is $\chi_1 \otimes \chi_2$.
When written in terms of the Techm\"uller character, we have $\omega^{k_1} \otimes \omega^{k_2} = \omega^{k_1+k_2}$.

Now we can define Gauss sums for finite fields, which depend upon a choice of an additive character and a multiplicative one. 
For finite fields, we will always use the additive character $\lambda$ defined above.
\begin{defn}\label{def:Gsum-FF}
The Gauss sum for a multiplicative character $\chi$ is denoted $S(\chi)$ or $S(\chi, \lambda)$ and is defined by \[S(\chi, \lambda) = \sum_{u \in \bbF_q^\times} \chi(u) \lambda(u).\]
\end{defn}
\noindent There are a few useful properties of Gauss sums that can be easily proven.
\begin{prop}
For a nontrivial multiplicative character $\chi$ on a finite field $\bbF_q$, we have that $|S(\chi)| = q^{1/2}$.
\end{prop}
\begin{prop}
For a multiplicative character $\chi$ on a finite field $\bbF_q$ with $q = p^n$, we have that $S(\chi^p) = S(\chi)$.
\end{prop}
This second proposition is because raising to the $p^\text{th}$ power only permutes the elements of the sum. 
For this reason we collect together the characters $\{\chi, \chi^{p}, \ldots, \chi^{p^{n-1}}\}$ together into a set we call the Frobenius orbit.
The converse theorem for Gauss sums of finite fields concerns separating the multiplicative characters of a finite field $\bbF_q$ into their Frobenius orbits.\\

However, first we separate out the characters that live in degenerate Frobenius orbits. 
These are the non-regular characters of $\bbF_q$ and are characterized by having $\chi = \chi^{p^k}$ for some $k \mid n-1$. 
If we write $\chi = \omega^k$, then $\chi$ is non-regular when there is some $k' \mid \frac{q-1}{p-1}$ with $k' \neq 1$ such that $k' \mid k$.
It suffices to check $k' = \frac{p^m - 1}{p-1}$ for $m \mid n$ and $m \neq 1$.
If a character is not non-regular, then it is regular. 
The regular characters are those with Frobenius orbits of size $n$ when $q = p^n$.\\

We can now state the conjectured converse theorem for Gauss sums of finite fields.
\begin{conj}[Nien]\label{conj:local-converse-FF}
Let $\chi_1$ and $\chi_2$ be two regular multiplicative characters of $\bbF_q$ with $q = p^n$ and $n$ prime. If \[S(\chi_1 \otimes \sigma) = S(\chi_2 \otimes \sigma)\text{, for all } \sigma \in \hat{\bbF_p^\times}\] then $\chi_1 = \chi_2^{p^i}$ for some integer $i$.
\end{conj}

For $n=2$; this becomes our theorem 
\begin{theorem}\label{thm:local-converse-FF}
Let $\chi_1$ and $\chi_2$ be two regular multiplicative characters of $\bbF_{p^2}$. If \[S(\chi_1 \otimes \sigma) = S(\chi_2 \otimes \sigma)\text{, for all } \sigma \in \hat{\bbF_p^\times}\] then $\chi_1 = \chi_2^{p}$ or $\chi_1 = \chi_2$.
\end{theorem}


\section{Stickelberger's Theorem}
Before we set out to prove our converse theorem for finite fields; it is simplest to reframe the problem as a question about the $p$-adic expansions of integers.
The tool that allows us to do this is Stickelberger's theorem which we shall introduce and prove in this section.
Working in the field $\bbQ(\mu_{q-1}, \mu_p)$, Stickelberger's theorem gives as a factorization of the ideal generated by $S(\omega^{-k})$ in terms of the the prime ideals over $p$ and the sum of the $p$-adic digits of $k$.
\\

As above; we will consider the case when $\bbF_q$ is a field of size $q = p^n$. 
We will let $\fp$ be a prime ideal in $\bbQ(\mu_{q-1})$ lying above $p$. 
Then we shall consider another extension $\bbQ(\mu_{q-1}, \mu_p)$ with $\fP$ being a prime ideal lying above $\fp$. Additionally; when it is useful for emphasis, we will let $\pi = \varepsilon - 1$
\\

Recall that the Teichm\"uller character generates $\hat{\bbF_q}$, so for any $\chi \in \hat{\bbF_q}$ we have $\chi = \omega^k$ for some $0 \leq k \leq q-1$, or equivalently, $\chi = \omega^{-k}$ for some $0 \leq k \leq q-1$, which will be more convenient when phrasing Stickelberger's theorem.
For such a $k$, we can write the $p$-adic expansion of $k$ as
\[k = k_0 + k_1 p + \cdots + k_{n-1}p^{n-1}\]
with $0 \leq k_i \leq p-1$. Then we can define the functions $s, \gamma : \bbZ \to \bbZ^+$ by 
\begin{align*}
s(k) &= k_0 + k_1 + \cdots + k_{n-1} \\
\gamma(k) &= k_0! k_1!  \cdots  k_{n-1}!
\end{align*}
for $0 \leq k < q-1$ and requiring that $s$ and $\gamma$ are $q-1$ periodic for other $k$.
\\

Our first step towards Stickelberger's theorem will be the following theorem which tells how many times one prime appears in the factorization of a Gauss sum.
\begin{theorem}\label{thm:Stickelberger}
For any integer $k$, we have the congruence 
\[\frac{S(\omega^{-k}, \varepsilon^\tr)}{(\varepsilon - 1)^{s(k)}} \equiv \frac{-1}{\gamma(k)} \mod \fP.\]
In particular, $\ord_\fP S(\omega^{-k}) = s(k)$.
\end{theorem}
To get the full factorization of $S(\omega^{-k})$, we need a few more definitions.\\

For $t \in \bbR$, we will let $0 \leq \langle t \rangle < 1$ be the representative of $t$ in $ \bbR / \bbZ$. 
Then let $G = \gal(\bbQ(\mu_m)/\bbQ) \cong (\bbZ / m\bbZ)^\times$ and let $\sigma_c \in G$ be defined by $\sigma_c : \zeta \mapsto \zeta^c$ and $\sigma_c|_{\mu_p}  = \id$; where $\zeta \in \mu_{q-1}$.
Then we make the following definition: 
\begin{defn}
We define the Stickelberger element in the group ring $\bbQ[G]$ as 
\[\theta(k,\fp) = \sum_{c \in (\bbZ/m\bbZ)^\times} \langle \frac{kc}{q-1}\rangle \sigma_c^{-1}\]
\end{defn}
Then we can get a full factorization of the ideal generated by $S(\omega^{-k})$. Using $\fa \sim \fb$ to mean that $\fa / \fb$ is the unit ideal, we have the following theorem.
\begin{theorem}
The factorization of the ideal generated by $S(\omega^{-k}, \varepsilon^\tr)$ is \[S(\omega^{-k}) \sim \fP^{(p-1)\theta(k, \fp)} \sim \fp^{\theta(k,\fp)}.\]
\end{theorem}



\section{The $n=2$ Local Converse Theorem for Finite Fields}
Recall that we would like to prove Conjecture \ref{conj:local-converse-FF} for the case of $n=2$.
By using the fact that $\exists k$ such that $\chi = \omega^k$, we can use Stickelberger's theorem rewrite the conjecture in the following way.

\begin{theorem}\label{thm:local-converse-FF-rewrite}
Suppose that $\alpha, \beta \in \bbZ/(p^2 -1)\bbZ$ with $p+1 \nmid \alpha, \beta$ and \[s(\alpha + k(p+1)) = s(\beta + k(p+1))\] for all $0 \leq k < p-1$. Then we have that $\alpha \equiv \beta \mod p^2 - 1$ or $\alpha \equiv p \beta \mod p^2 -1$.
\end{theorem}

\begin{prop}
Theorem \ref{thm:local-converse-FF} and Theorem \ref{thm:local-converse-FF-rewrite} are equivalent.
\end{prop}

\begin{proof}
Suppose the conditions of Theorem \ref{thm:local-converse-FF} hold. 
If we write $\chi_1 = \omega^\alpha$ and $\chi_2 = \omega^\beta$ and use the fact that all the twists can be written as $\sigma = \omega^{k (p+1)}$ for $0 \leq k < p-1$; then we are claiming that $S(\omega^{\alpha + k(p+1)}) = S(\omega^{\beta+k(p+1)})$ for all $0 \leq k < p-1$. 
However, by Theorem \ref{thm:Stickelberger} if $S(\omega^{\alpha + k(p+1)}) = S(\omega^{\beta+k(p+1)})$ for all $0 \leq k < p-1$ then $s(\omega^{\alpha + k(p+1)}) = s(\omega^{\beta+k(p+1)})$ for all $0 \leq k < p-1$. 
This implies the conditions of Theorem \ref{thm:local-converse-FF-rewrite}.
\\

In the other direction, suppose that the conditions of Theorem \ref{thm:local-converse-FF-rewrite} hold. 
Then $p+1 \nmid \alpha, \beta$ ensures that $\chi_1 = \omega^\alpha$ and $\chi_2 = \omega^\beta$ are regular characters of $\bbF_{p^2}$. Then $s(\alpha + k(p+1)) = s(\beta + k(p+1))$ for all $0 \leq k < p-1$ only provides that $\ord_{\fP} S(\chi_1 \otimes \sigma) = \ord_{\fP} S(\chi_2 \otimes \sigma)$ for all twists $\sigma$. \textcolor{red}{Rewrite this! Have actual proof on paper.}

\end{proof}

Now that we only need to prove Theorem \ref{thm:local-converse-FF-rewrite}, we shall prove the $n=2$ local converse theorem for finite fields.

\begin{proof}
Note that we only need to consider the case of fixed $\alpha$ and can vary $\beta$ to find those that satisfy the equality of gauss sums. 
For the case $n=2$ we have the $\omega^{-\alpha}$ is regular as long as $(p+1)\nmid \alpha$.

In this case we find that \[S(\omega^{-(\alpha + \hat{k})}) = S(\omega^{-(\beta + \hat{k})})\] for all $0 \leq k < p-1$ clearly implies that $S(\omega^{-(\alpha+\hat{k})})$ and $S(\omega^{-(\beta + \hat{k})})$ have the same factorization for all $0 \leq k < p-1$ which by Stickelberger's theorem, is equivalent to the condition that $s(\alpha + k(p+1)) = s(\beta + k(p+1))$ for all $0 \leq k < p-1$. 
We would like to show that this condition implies that $\alpha \equiv p \beta \mod p^2 -1$ or $\alpha \equiv \beta \mod p^2 - 1$. 
Another way of stating this is that if $s(\alpha + k(p+1)) = s(\beta+k(p+1))$ for all $0 \leq k < p-1$ then $\alpha$ and $\beta$ are in the same Frobenius orbit.

We will first do a few simple cases, and then consider the general case.
We will let $s(\alpha) = t$ be the sum of the digits of $\alpha$ in base $p$ and note that when we split into cases for different $t$, we are splitting into the cases where $S(\omega^\alpha)$ and  $S(\omega^\beta)$ have the same factorization without considering any twists.
\\

\subsection{$t = 1$:} If $t = 1 = s(\alpha)$, then $\alpha = 1$ and $\alpha = p$ are the only possibilities. 
These are both clearly in the same Frobenius orbit $\{1,p\}$ and no twisting is necessary here.
\\

\subsection{$t = 2$:} If $s(\alpha) = t = 2$, then the possibilities are $\alpha = 2, 1+p, 2p$. 
We can rule out $1+p$ because $1+p \mid 1+p$ means that $\alpha = 1+p$ does not give a regular character. 
Then $\alpha = 2, 2p$ are clearly in the same Frobenius orbit $\{2,2p\}$ and again, no twisting is necesary. 
Now let us look at a case where twisting is necessary.
\\

\subsection{$t = 3$:} If $s(\alpha) = t= 3$, then the possibilites are $\alpha = 3,2+p,1+2p,3p$. 
We have that the two Frobenius orbits are $\{3,3p\}$ and $\{2+p, 1+2p\}$ and we would like to show that taking the digit sum of some twists will let us distinguish between them when digit sums of the $\alpha$ alone do not.

Let us consider what happens when $k = p-3$ for $p > 3$ and compute $s(\alpha + k(p+1))$ for one $\alpha$ in each Frobenius orbit. 
We find that \[s(3 + (p-3)(p+1)) = s(((p-3)+3)+(p-3)p) = s((p-3+1)p) = s((p-2)p) = p-2.\] 
On the other hand, we have that \[s((p+2)+(p-3)(p+1)) = ((p-3)+1) + ((p-3)+2) =2p-3\]
Since $p-2 \neq 2p-3$ for all $p \neq 1$ (and therefore all primes) we have that twisting by $k = p-3$ enables us to differentiate between the two Frobenius orbits having $t = 3$ for $p > 3$. 
For $p = 2,3$, the case of $t = 3$ is not a concern because $t =3 > p-1$ which is not under consideration right now. 
We will deal with the case of $t > p-1$ later.
\\

Now that we have looked at a few specific cases, let us consider the general $0 \leq t \leq p-1$. 
Then we having the following list of possibilites of $\alpha$: \[\alpha = t, (t-1) + p, (t-2) +2 p, \cdots 2 + (t-2)p , 1 + (t-1)p, tp.\] 
These possibilities can be indexed by some $0 \leq i \leq t$ and then we find that $\{ (t-i) + i p \mid 0 \leq i \leq t\}$ is our set of possibilites. 
It will be simplest to split into two cases, the even case and the odd case.
We will let $r = \left\lfloor \frac{t}{2} \right\rfloor$ in both cases.
Then we have that $t = 2r$ in the even case and $t = 2r+1$ in the odd case.
The main difference is that when $t$ is even, $\alpha = r + rp$ is a possible character that we eliminate by restricting to only regular characters.

\subsection{$t = 2r$:}
We can split into the Frobenius orbits that we wish to distinguish between. 
They pair up nicely and we get that the orbits are \[\{t, tp\}, \{(t-1)+p, 1+(t-1)p\}, \cdots, \{(r-1) + (r+1) p, (r+1) + (r-1) p\}.\] 
Written more generally, the orbits for $s(\alpha) = t$ look like $\{(t-i) + ip, i + (t-i)p\}$ for integers $0 \leq i < r $, where $i=r$ would give the degenerate orbit/non-regular character and $i > r$ just repeats the list again. 
Our goal is then to distinguish between these orbits by looking at some twists of these values.
I claim that it is sufficient to consider $k = p - \ell$ for $2 \leq \ell \leq r < t$.
We are always considering $t < p$, so $k = p - \ell \geq p - t > 0$.
\\

\textcolor{red}{NOTE: It seems like $2 \leq \ell \leq r+1$ has the same distinguishing power as $r+2 \leq \ell \leq 2r=t$?}
\\

We shall proceed by direct computation for $t > 3$.
First, let us consider the two extremes. 
We have that with $\ell = 2$ the twists have the following digit sum:
\begin{align*}
 s((t-i) + (i)p + (p - 2)(p+1)) &= s((p + t - i - 2) + (p + i -2)p) \\ 
 &= \begin{cases}
   s((t-2) + (p-2+1)p) & i = 0 \\
   s((t-i-2) + (p+i-1)p)& i \neq 0
 \end{cases} \\
 &= \begin{cases}
   p+t - 3 & i = 0 \\
   s((t-i-1)+(i-1)p) & i \neq 0
 \end{cases} \\
 s((t-i) + ip + (p - 2)(p+1)) &= \begin{cases}
   p+t - 3 & i = 0 \\
   t-2 & i \neq 0
 \end{cases}
\end{align*}
All we have done here is moved a $p$ from one coefficient to the other other and turned it into a carry of $1$ whenever a coefficient was $\geq p$. 
Note that this assumes that $t \geq 2$ to have that $p+t-2 \geq p$, but the only even $t < 2$ is $t =0$ which gives only the trivial character.
\\

On the other hand, let's see what happens for the other extreme. 
When $\ell = r$, we have the following digit sums:
\begin{align*}
 s((t-i) + (i)p + (p - r)(p+1)) &= s((p + t- i - r) +(p + i -r) p) \\ 
 &=s((p + r - i) + (p -r + i)p) \\
 &= \begin{cases}
s((r-i) + (p+i-r+1)p) & i < r
 \end{cases} \\
 &= \begin{cases}
s((r-i) + (p+i-r+1)p) & i < r-1 \\
 s((r-i+1) + (i-r+1)p) & i = r-1
 \end{cases} \\
  s((2r-i) + ip + (p - r)(p+1)) &= \begin{cases}
p+1 & i < r-1 \\
2 & i = r-1
 \end{cases}
\end{align*}
Here, we use $i < r$, but we already mentioned above that this is just choosing the element of the orbit with the smaller coefficient for $p$ and is not actually limiting the cases computed. 
We also implicitly assume $r < p$, but this is fine, because for now we are dealing with the case of $t < p$ and $t = 2r$ so $ r < p/2$.
\\

There are two interesting things we see in these computations. 
The first is that depending on what $\ell$ is, twisting by $k = p - \ell$ seems to split into two separate cases which have distinct digit sums (and note that $p+1 \neq 2$ and $p+2r-3 \neq 2r-2$ for all primes $p$). 
The second is that twisting with $\ell = 2$, the smallest $\ell$ we can twist by distinguishes the orbit with $i = 0$, the smallest orbit, and likewise, $\ell = r$, the largest $\ell$ we choose to twist by distinguishes the orbit with $i = r-1$, the largest $i$ necessary to consider. 
It is too much to hope that each $\ell$ we use to twist will distinguish a single orbit, but we will see that each $\ell$ separates the orbits in to two classes, and taken together, they can be used to distinguish all the orbits.
\\

The general case here will just be to twist by $k = p- \ell$ with $2 \leq \ell \leq r$. 
Then we have the following possible digit sums:
\begin{align*}
s((2r-i) + (i)p + (p - \ell)(p+1)) &= s((p + 2r - i - \ell) + (p + i -\ell)p) \\ 
&=\begin{cases}
  s((2r - i - \ell) + (p + 1 + i -\ell)p) & 2r -i  \geq \ell \\
  s((p + 2r - i - \ell) + (p + i -\ell)p) & 2r - i <  \ell 
\end{cases}\\
&=\begin{cases}
  s((2r + 1 - i - \ell) + ( 1 + i -\ell)p) & 2r -i  \geq \ell \text{ and } 1+i \geq \ell \\
  s((2r - i - \ell) + (p + 1 + i -\ell)p) & 2r -i  \geq \ell  \text{ and } 1 + i < \ell \\
  s((p + 2r - i - \ell) + (p + i -\ell)p) & 2r - i <  \ell \text{ and } i < \ell \\
  s((p + 2r +1 - i - \ell) + (1 + i -\ell)p) & 2r - i <  \ell \text{ and } i \geq \ell
\end{cases} \\
s((2r-i)+(i)p + (p-\ell)(p+1))
&= \begin{cases}
  p+2r-2\ell + 1 & i \leq \ell - 2 \\
  2r-2\ell + 2 & i > \ell -2
\end{cases}
\end{align*}
\\

We can reduce from the four cases down to two for a few reasons.
First, note that $1 + i \geq \ell \implies i \geq \ell - i \implies i > \ell -2$ for integers $i$.
Additionally, $2r -i \geq \ell$ is the only possibly case because $i < r$ and $\ell \leq r$ so the smallest possible $2r-i$ is bigger than the biggest possible $\ell$.
Finally, in the $2r - i \geq \ell$ and $1 + i < \ell$ case, $1 + i < \ell \implies i < \ell -1 \implies i \leq \ell -2$ for integers $i$; and $i \leq \ell -2$ means that $p + 1 + i - \ell \leq p-1$ so no further reduction is necessary.

\subsection{$t = 2r+1$:}
For the case when $t$ is odd, much of the same analysis follows through in the same way with a few $+1$'s scattered throughout.
Again, we can split into the Frobenius orbits that we wish to distinguish between. 
They pair up nicely, this time with no degeneracy and we get that the orbits are \[\{t, tp\}, \{(t-1)+p, 1+(t-1)p\}, \cdots, \{r + (r+1) p, (r+1) + r p\}.\] 
Written more generally, the orbits for $s(\alpha) = t$ look like $\{(t-i) + ip, i + (t-i)p\}$ for integers $0 \leq i \leq r $ where $i > r$ just repeats the same list again.
Our goal is then to distinguish between these orbits by looking at some twists of these values.
I claim that it is sufficient to consider $k = p - \ell$ for $2 \leq \ell \leq r + 1 < t$ (this last equality only true for odd $t > 1$).
We are always considering $t < p$, so $k = p - \ell \geq p - t > 0$.
\\

We shall proceed by direct computation for $t \geq 3$, and deal only with the general case.
The general case here will just be to twist by $k = p- \ell$ with $2 \leq \ell \leq r+1$. 
Then we have the following possible digit sums:
\begin{align*}
s((2r+1-i) + ip + (p - \ell)(p+1)) &= s((p + 2r+1 - i - \ell) + (p + i -\ell)p) \\ 
&=\begin{cases}
  s((2r +1- i - \ell) + (p + 1 + i -\ell)p) & 2r+1-i  \geq \ell \\
\end{cases}\\
&=\begin{cases}
  s((2r + 1 - i - \ell) + ( 1 + i -\ell)p) &  1+i \geq \ell \\
  s((2r +1- i - \ell) + (p + 1 + i -\ell)p) & 1 + i < \ell \\
\end{cases} \\
s((2r+1-i)+ip + (p-\ell)(p+1))
&= \begin{cases}
  p+2r+1-2\ell + 1 & i \leq \ell - 2 \\
  2r+1-2\ell + 2 & i > \ell -2
\end{cases}
\end{align*}
\\

As before, we only have the $2r+1-e \geq \ell$ case because $i < r$ and $\ell \leq r+1$ so the smallest possible $2r+1-i$ is bigger than the biggest possible $\ell$.
Then, note that $1 + i \geq \ell \implies i \geq \ell - i \implies i > \ell -2$ for integers $i$.
Finally, in the $1 + i < \ell$ case, $1 + i < \ell \implies i < \ell -1 \implies i \leq \ell -2$ for integers $i$; and $i \leq \ell -2$ means that $p + 1 + i - \ell \leq p-1$ so no further reduction is necessary.
\\


\textcolor{red}{TODO: It would be cleaner to separate out the equivalence for $s((t-i)+ip+(p-\ell)(p+1))$ into it's own lemma with the even and odd casework moved into that portion, or even two lemmas with a joint corollary.}
\\

\subsection{rest of the proof}
Now that we have (what will be a lemma) our main idea:
\[ s((t-i)+ip+(p-\ell)(p+1)) = \begin{cases} p + t-2\ell+1 & i \leq \ell - 2 \\ t - 2 \ell + 2 & i > \ell -2\end{cases},\]
we can finish up the rest of the proof.
The idea here is that as we twist by more $\ell$'s, we distinguish one additional orbit each time.
Recall for the $\ell = 2$ twist we had that the $i = 0$ orbit was distinguished from all the others because it has a different digit sum after the twist.
Then, for the $\ell = 3$ twist, we find that $i=0,1$ are separated from $i > 1$, but since $i =0$ is already distinguished, we do not have to distinguish it agin, and so we have distinguished $i = 1$.
It is clear that the $\ell$ twist will separate the orbits into those with $i \leq \ell - 2$ and those with $i > \ell -2$.
Then we have that if we consider all the twists $2 \leq \ell \leq m \leq \left\lceil \frac{t}{2} \right\rceil$, then we have fully distinguished all the $i \leq m-2$ but cannot tell apart any of the $m-2 < i < \left\lceil \frac{t}{2} \right\rfloor$ orbits.
With $m = \left\lceil \frac{t}{2} \right\rceil$, we are able to fully distinguish all of the orbits with a given $t < p$.
\\
\textcolor{red}{TODO: Turn this bit into a lemma.}
\\
\textcolor{red}{TODO: Make this use $t$ instead of $\alpha$.}
Finally, we need to show that this still holds for $p \leq \alpha \leq 2p - 2$. 
Note that we have that $s(p^2 - \alpha) = 2p-1  - s(\alpha)$ because if $\alpha = ip+j$ then \[s(p^2 -\alpha) = s((p-i)p -j) = s((p -i - 1)p + (p-j) )= 2p-1 - (i+j) = 2p - 1 - s(\alpha).\]
\end{proof}

This means that if $s(\alpha) \geq p$, then $s(p^2 - \alpha) = s(-\alpha) < p$, which will allow us to finish distinguishing between the different $\alpha$'s. 
Specifically, suppose that $\omega^{-\alpha}$ and $\omega^{-\beta}$ satisfy the conditions of the conjecture and $s(\alpha) \geq p$. 
Then $S(\omega^{-\alpha}) = S(\omega^{-\beta}) \implies s(\alpha) = s(\beta) \geq p$. 
Then we also have that $s(p^2 - \alpha) = s(p^2 - \beta) < p$. 
By our lemma \textcolor{red}{??? (because $S(\omega^{-\alpha}) = \omega^{\alpha}(-1)S(\omega^{\alpha})$)}, $S(\omega^\alpha) = S(\omega^\beta)$ if and only if $S(\omega^{-\alpha}) = S(\omega^{-\beta})$ and so using the prior part of the theorem to distinguish between $p^2 - \alpha$ and $p^2 - \beta$, we can also distinguish between $\alpha$ and $\beta$ since twisting $-\alpha$ by $k$ is the same as twisting $\alpha$ by $-k$ 

\endinput