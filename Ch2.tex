\chapter{Local Converse Theorem for Finite Fields}	%Chapter title
\begin{itemize}
\item Finite Fields 
\begin{itemize}
\item additive characters of $\hat{\bbF}_p$
\item multiplicative characters of $\mathbb{F}_p^\times$ and $\mathbb{F}_{p^n}^\times$
\item Trace $\mathbb{F}_{p^n}^\times \to \mathbb{F}_{p}^\times$
\item Regular characters
\end{itemize}
\item Gauss Sums 
\item Teichmuller Characters 
\item Stickelberger Theorem 
\begin{itemize}
\item A description of the prime ideals involved?
\item $p$-adic expansions and how they relate to Gauss sums on a first order of approximation, especially if Gross-Koblitz gets included
\end{itemize}
\item Gross-Koblitz Formula if necessary 
\begin{itemize}
\item If I include this, then I need $p$-adic gamma function
\end{itemize}
\item Restate the converse theorem in terms of $s(k)$ instead of $S(\omega^{-k})$.
\item Do I need a $1$D analog converse for characters of $\mathbb{F}_p^\times$ or $\mathbb{F}_q^\times$ 
\item $2$D analog converse for characters of $\mathbb{F}_{p^2}^\times$ twisted by characters of $\mathbb{F}_{p}^\times$ 
\item If necessary, converse for characters of $\mathbb{F}_{q^2}^\times$ twisted by characters of $\mathbb{F}_{q}^\times$; need to look into a proof of this.
\item If possible, converse for characters of $\mathbb{F}_{p^3}^\times$ twisted by characters of $\mathbb{F}_{p}^\times$ 
\end{itemize}

\section{Gauss Sums for Finite Fields}
\noindent We will use the following notation in this chapter.\\
$\bbF_p$ will be the finite field with $p$ elements.\\
$\bbF_q$ will be the finite field with $q = p^n$ elements for some $n \geq 1$.\\
For any finite field $\bbF_q$, $\bbF_q^\times$ will be the multiplicative unit group of the field.\\
$\varepsilon$ will be a primitive $p^\text{th}$ root of unity. For $\varepsilon \in \bbC$, we use $\varepsilon = e^{2 \pi i /p}$.\\
$\mu_N$ will be the group of $N^\text{th}$ roots of unity.\\

We will be dealing only with complex representations of finite fields.
Since $\bbF_q$ is a field, all representations are $1$ dimensional characters.
An additive character of $\bbF_q$ is a function $\xi : \bbF_q \to \bbC^\times$ with $\xi(a + b) = \xi(a)\xi(b)$.
A multiplicative character of $\bbF$ is a function $\chi : \bbF_q^\times \to \bbC^\times$ with $\chi(ab) = \chi(a)\chi(b)$.
If we ever need to evaluate a multiplicative character $\chi$ at $0 \not \in \bbF_q^\times$ then we use $\chi(0) = 0$. \\

The two most important characters in this setting are the additive character defined by the trace and the multiplicative character called the Teichm\"uller character. 
The primary trace we are interested in is the trace for the extension $\bbF_q / \bbF_p$ which is defined in the following way.
%rephrase this to remove defn environment; only want defn env when will refer to it later
\begin{defn}
The trace from $\bbF_q / \bbF_p$ for $q = p^n$ is defined by $\tr : \bbF_q \to \bbF_p$
 \[ \tr : x \mapsto \sum_{i=0}^{n-1} x^{p^i} = x + x^p + x^{p^2} + \cdots + x^{p^{n-1}}\]
\end{defn}
\noindent Then we can define an additive character $\lambda : \bbF_q \to \mu_p$ by $\lambda(x) = \epsilon^{\tr(x)}$.
It takes a bit more to define the Teichm\"uller character.\\

Consider the field $\bbQ(\mu_{q-1})$ and let $\fp$ be the prime ideal lying over $p$.
Then the residue field of $\bbQ(\mu_{q-1})$ mod $\fp$ is isomoprhic to $\mu_{q-1}$ which in turn is isomorphic to $\bbF_q^\times$.
\begin{defn}\label{Teichmuller}
We define the Teichm\"uller character by $\omega : \bbF_q^\times \to \mu_{q-1}$ by $\omega (u) \equiv  u \mod \fp$.
\end{defn}
\noindent This character generates the character group of $\bbF_q^\times$; so for all multiplicative characters $\chi$ of $\bbF_q^\times$ there is some integer $k$ such that $\chi = \omega^k$.\\




\endinput