\chapter{Local Converse Theorem for Finite Fields}	%Chapter title
\begin{itemize}
\item Finite Fields 
\begin{itemize}
\item additive characters of $\hat{\bbF}_p$
\item multiplicative characters of $\mathbb{F}_p^\times$ and $\mathbb{F}_{p^n}^\times$
\item Trace $\mathbb{F}_{p^n}^\times \to \mathbb{F}_{p}^\times$
\item Regular characters
\end{itemize}
\item Gauss Sums 
\item Teichmuller Characters 
\item Stickelberger Theorem 
\begin{itemize}
\item A description of the prime ideals involved?
\item $p$-adic expansions and how they relate to Gauss sums on a first order of approximation, especially if Gross-Koblitz gets included
\end{itemize}
\item Gross-Koblitz Formula if necessary 
\begin{itemize}
\item If I include this, then I need $p$-adic gamma function
\end{itemize}
\item Restate the converse theorem in terms of $s(k)$ instead of $S(\omega^{-k})$.
\item Do I need a $1$D analog converse for characters of $\mathbb{F}_p^\times$ or $\mathbb{F}_q^\times$ 
\item $2$D analog converse for characters of $\mathbb{F}_{p^2}^\times$ twisted by characters of $\mathbb{F}_{p}^\times$ 
\item If necessary, converse for characters of $\mathbb{F}_{q^2}^\times$ twisted by characters of $\mathbb{F}_{q}^\times$; need to look into a proof of this.
\item If possible, converse for characters of $\mathbb{F}_{p^3}^\times$ twisted by characters of $\mathbb{F}_{p}^\times$ 
\end{itemize}

\section{Gauss Sums for Finite Fields}
\noindent We will use the following notation in this chapter.\\
$\bbF_p$ will be the finite field with $p$ elements.\\
$\bbF_q$ will be the finite field with $q = p^n$ elements for some $n \geq 1$.\\
For any finite field $\bbF_q$, $\bbF_q^\times$ will be the multiplicative unit group of the field.\\
$\varepsilon$ will be a primitive $p^\text{th}$ root of unity. For $\varepsilon \in \bbC$, we use $\varepsilon = e^{2 \pi i /p}$.\\
$\mu_N$ will be the group of $N^\text{th}$ roots of unity.\\

We will be dealing only with complex representations of finite fields.
Since $\bbF_q$ is a field, all representations are $1$ dimensional characters.
An additive character of $\bbF_q$ is a function $\xi : \bbF_q \to \bbC^\times$ with $\xi(a + b) = \xi(a)\xi(b)$.
A multiplicative character of $\bbF$ is a function $\chi : \bbF_q^\times \to \bbC^\times$ with $\chi(ab) = \chi(a)\chi(b)$.
If we ever need to evaluate a multiplicative character $\chi$ at $0 \not \in \bbF_q^\times$ then we use $\chi(0) = 0$. \\

The two most important characters in this setting are the additive character defined by the trace and the multiplicative character called the Teichm\"uller character. 
The primary trace we are interested in is the trace for the extension $\bbF_q / \bbF_p$ which is defined in the following way.
%rephrase this to remove defn environment; only want defn env when will refer to it later
\begin{defn}
The trace from $\bbF_q / \bbF_p$ for $q = p^n$ is defined by $\tr : \bbF_q \to \bbF_p$
 \[ \tr : x \mapsto \sum_{i=0}^{n-1} x^{p^i} = x + x^p + x^{p^2} + \cdots + x^{p^{n-1}}.\]
\end{defn}
\noindent Then we can define an additive character $\lambda : \bbF_q \to \mu_p$ by $\lambda(x) = \epsilon^{\tr(x)}$.
It takes a bit more to define the Teichm\"uller character.\\

Consider the field $\bbQ(\mu_{q-1})$ and let $\fp$ be the prime ideal lying over $p$.
Then the residue field of $\bbQ(\mu_{q-1})$ mod $\fp$ is isomorphic to $\mu_{q-1}$ which in turn is isomorphic to $\bbF_q^\times$.
\begin{defn}\label{Teichmuller}
We define the Teichm\"uller character $\omega : \bbF_q^\times \to \mu_{q-1}$ by \[\omega (u) \equiv  u \mod \fp.\]
\end{defn}
\noindent This character generates the character group of $\bbF_q^\times$; so for all multiplicative characters $\chi$ of $\bbF_q^\times$ there is some integer $k$ such that $\chi = \omega^k$.\\
When we have two multiplicative character $\chi_1$ and $\chi_2$, we say that the twist of $\chi_1$ by $\chi_2$ is $\chi_1 \otimes \chi_2$.
When written in terms of the Techm\"uller character, we have $\omega^{k_1} \otimes \omega^{k_2} = \omega^{k_1+k_2}$.

Now we can define Gauss sums for finite fields, which depend upon a choice of an additive character and a multiplicative one. 
For finite fields, we will always use the additive character $\lambda$ defined above.
\begin{defn}\label{GSumFF}
The Gauss sum for a multiplicative character $\chi$ is denoted $S(\chi)$ or $S(\chi, \lambda)$ and is defined by \[S(\chi, \lambda) = \sum_{u \in \bbF_q^\times} \chi(u) \lambda(u).\]
\end{defn}
\noindent There are a few useful properties of Gauss sums that can be easily proven.
\begin{prop}
For a nontrivial multiplicative character $\chi$ on a finite field $\bbF_q$, we have that $|S(\chi)| = q^{1/2}$.
\end{prop}
\begin{prop}
For a multiplicative character $\chi$ on a finite field $\bbF_q$ with $q = p^n$, we have that $S(\chi^p) = S(\chi)$.
\end{prop}
This second proposition is because raising to the $p^\text{th}$ power only permutes the elements of the sum. 
For this reason we collect together the characters $\{\chi, \chi^{p}, \ldots, \chi^{p^{n-1}}\}$ together into a set we call the Frobenius orbit.
The converse theorem for Gauss sums of finite fields concerns separating the multiplicative characters of a finite field $\bbF_q$ into their Frobenius orbits.\\

However, first we separate out the characters that live in degenerate Frobenius orbits. 
These are the non-regular characters of $\bbF_q$ and are characterized by having $\chi = \chi^{p^k}$ for some $k \mid n-1$. 
If we write $\chi = \omega^k$, then $\chi$ is non-regular when there is some $k' \mid \frac{q-1}{p-1}$ with $k' \neq 1$ such that $k' \mid k$.
It suffices to check $k' = \frac{p^m - 1}{p-1}$ for $m \mid n$ and $m \neq 1$.
If a character is not non-regular, then it is regular. 
The regular characters are those with Frobenius orbits of size $n$ when $q = p^n$.\\

We can now state the conjectured converse theorem for Gauss sums of finite fields.
\begin{conj}[Nien]
Let $\chi_1$ and $\chi_2$ be two regular multiplicative characters of $\bbF_q$ with $q = p^n$ and $n$ prime. If \[S(\chi_1 \otimes \sigma) = S(\chi_2 \otimes \sigma)\text{, for all } \sigma \in \hat{\bbF_p}\] then $\chi_1 = \chi_2^{p^i}$ for some integer $i$.
\end{conj}
\noindent \textcolor{red}{What we end up proving is the case of $n = 2$.}


\section{Stickelberger's Theorem}
Before we set out to prove our converse theorem for finite fields; it is simplest to reframe the problem as a question about the $p$-adic expansions of integers.
The tool that allows us to do this is Stickelberger's theorem which we shall introduce and prove in this section.

\endinput