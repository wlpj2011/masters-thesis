\documentclass[presentation]{beamer}
\usepackage{wlpjbase}

\setbeamercovered{invisible}

%\documentclass[handout, t]{beamer}
\usetheme{default}
\usecolortheme{rose}
\usetheme{Singapore}
\setbeamertemplate{caption}[numbered]
%\useinnertheme[shadow]{rounded}
\useoutertheme{default}
\usepackage{tikz-cd,float}
\newcommand{\highlight}[1]{%
  \colorbox{blue!25}{$\displaystyle#1$}}
\usetikzlibrary{arrows.meta,calc}
\usepackage{graphicx}
\makeatletter
\newcommand*\bigcdot{\mathpalette\bigcdot@{.5}}
\newcommand*\bigcdot@[2]{\mathbin{\vcenter{\hbox{\scalebox{#2}{$\m@th#1\bullet$}}}}}
\makeatother
\usepackage{amsmath,amsfonts,enumerate}
\usetikzlibrary{positioning, shapes.geometric}
%\usepackage{tkz-graph}
%\usetikzlibrary{knots,shapes,arrows,angles,quotes,decorations.markings}
%\usepackage{pgfplots}
%\usepackage{subfigure}
%\usepackage{hyperref}
%\usepackage[linesnumbered,ruled]{algorithm2e}
%\usepackage{animate}
%\usepackage{amscd}
%\usepackage{verbatim}
%\usepackage{tkz-graph}
%\usetikzlibrary{calc}
%\usepackage{xmpmulti}
%\usepackage{caption}	
%\usepackage{amscd}	

\usetikzlibrary{plotmarks}	% Lets you have different shaped nodes as the trees
%\usepackage{stix}  %Let's you use \circlerighthalfblack for the node that is half-white & half-black
%\usepackage{mathdots} 


%\newcommand{\tikzcircle}[2][red,fill=red]{\tikz[baseline=-0.5ex]\draw[#1,radius=#2] (0,0) circle ;}%


\newcommand{\tcr}[1]{\textcolor{red}{#1}}
\newcommand{\tcb}[1]{\textcolor{blue}{#1}}
\newcommand{\tcm}[1]{\textcolor{magenta}{#1}}
\newcommand{\tcg}[1]{\textcolor{green}{#1}}
\newcommand{\mP}{\mathcal{P}}
\newcommand{\ol}{\overline}
\usepackage[most]{tcolorbox}
%\setbeamertemplate{navigation symbols}{} 
\usepackage[english]{babel}  % or whatever
\usepackage[latin1]{inputenc} % or whatever
\usepackage{times}
\usepackage{color}
\usepackage{verbatim}

\usepackage[T1]{fontenc}  
\usepackage{booktabs}

\newcommand{\p}{}  % for pausing itemized lists
%\usepackage{cancel}
%\usepackage{wrapfig,hyperref}
%\usepackage{authblk}
%\usepackage{mathtools}
%\usepackage{fullpage}
%\usepackage{subfig}
%\usepackage{color}
%\usepackage[normalem]{ulem}
%\usepackage{tikz}
%\usepackage{hyperref}
\hypersetup{
    colorlinks=true,
    linkcolor=blue,
    filecolor=magenta,
    urlcolor=cyan,
}


\setlength{\tabcolsep}{3pt} % changes spacing between columns of a table


\title[] % (optional, use only with long paper titles)
{Stickelberger's Theorem and Factorization of Gauss Sums
}


\author[WJ]
{William Johnson
}

\institute[2024] 
{
 University of Maine
}
\date[Thesis 2024]
{February 27, 2024}
\begin{document}

\frame{\titlepage}
%\begin{frame}
 % \titlepage
%\end{frame}


\begin{frame}{Introduction}
\bigskip
In this presentation I will discuss:
\begin{itemize}
    \item the basics of the theory of Gauss sums for finite fields
    
    \pause
    
    \item a sketch of the proof of Stickelberger's theorem 
    \pause 
    
    \item an application of Stickelberger's theorem to factoring Gauss sums
    \pause 
\end{itemize}
\bigskip
This presentation includes foundational material for my thesis advised by Gil Moss. 
\end{frame}


\begin{frame}{Finite Fields}
We let $\bbF_p$ be a finite field with $p$ elements. This is typically presented as $(\bbZ / p \bbZ, +, \times)$.\\

\pause
\vspace{0.5cm}

Then we have the finite fields $\bbF_q$ where $q = p^n$ for some $n$. 

\pause
\vspace{0.5cm}

These are $\bbF_p[X]/\langle f \rangle$ for some irreducible degree $n$ polynomial $f \in \bbF_p[X]$. \\

\pause
\vspace{0.5cm}

This is not the same as the ring $(\bbZ/ q \bbZ, +, \times)$. For $q = p^n$, $(\bbZ/q\bbZ)^\times$ has $\phi(q) = \phi(p^n) = (p-1)p^{n-1}$ elements; $(\bbF_q)^\times$ has $q-1 = p^n - 1$ elements.
\end{frame}

\begin{frame}{Additive Characters}
An \textbf{additive character} of $\bbF_q$ is a homomorphism from the group $(\bbF_q, +)$ to $\bbC^\times$. \\

\pause 
\vspace{0.5cm}

We use the trace map $\tr : \bbF_q \to \bbF_p$ to define a canonical additive character. For $q = p^n$ the trace is \[\tr x \mapsto \sum_{i=0}^{n-1} x^{p^i}.\]\\

\pause 
\vspace{0.5cm}

We then define an additive character $\lambda : \bbF_q \to \bbC^\times$ by $\lambda : x \mapsto \varepsilon^{\tr x}$ where $\varepsilon = e^{2 \pi i /p}$. 

\end{frame}

\begin{frame}{Multiplicative Characters}
A \textbf{multiplicative character} of $\bbF_q$ is a homomorphism from the group $(\bbF_q^\times, \times)$ to $\bbC^\times$. 
\\

\pause
\vspace{0.5cm}

$(\bbF_q^\times, \times)$ is a cyclic group isomorphic to $\bbZ/(q - 1)\bbZ$ and so is $\widehat{\bbF_q^\times}$. So any multiplicative character $\chi = \omega^k$ for some $0 \leq k < q-1$.
\\

\pause
\vspace{0.5cm}

There is some multiplicative character $\omega$ that generates the multiplicative character group of $\bbF_q$.

\end{frame}

\begin{frame}{Cyclotomic Extensions of $\bbQ$}
An additive character of $\bbF_q$ takes values in $\mu_p$ and a multipicative character of $\bbF_q$ takes values in $\mu_{q-1}$, where $\mu_m$ means the set of $m^\text{th}$ roots of unity in $\bbC^\times$.\\

We consider the fields $\bbQ(\mu_p)$ and $\bbQ(\mu_{q-1})$.
\\
\pause 
\vspace{0.5cm}

When we take sums and products of the values of an additive character and a multiplicative character, this places us in the mutual extension $\bbQ(\mu_p, \mu_{q-1})$. 
\\
\pause
\vspace{0.5cm}
This is where the Gauss sums for $\bbF_q$ will live.


\end{frame}

\begin{frame}{Prime Ideals in Cyclotomic Extensions of $\bbQ$}

We are mainly interested in the primes lying above $\langle p \rangle = p\bbZ $ in our extensions of interest. We let $\fp \subset \cO_{\bbQ(\mu_{q-1})}$ be a prime lying above $p\bbZ$ and $\fP \subset \cO_{\bbQ(\mu_p, \mu_{q-1})}$ lying above $\fp$. ($\cO_K$ is the ring of integers in the field $K$).
\pause

\[\begin{tikzcd}[ampersand replacement=\&, column sep=small]
	\& {\mathbb{Q}(\mu_p,\mu_{q-1})} \&\&\& {\mathfrak{P}} \\
	{\mathbb{Q}(\mu_p)} \&\& {\mathbb{Q}(\mu_{q-1})} \& {p\mathcal{O}_{\mathbb{Q}(\mu_p)}} \&\& {\mathfrak{p}} \\
	\& {\mathbb{Q}} \&\&\& {p\bbZ}
	\arrow[no head, from=3-2, to=2-3]
	\arrow[no head, from=3-2, to=2-1]
	\arrow[no head, from=2-3, to=1-2]
	\arrow[no head, from=2-1, to=1-2]
	\arrow[no head, from=3-5, to=2-6]
	\arrow[no head, from=2-6, to=1-5]
	\arrow[no head, from=3-5, to=2-4]
	\arrow[no head, from=2-4, to=1-5]
\end{tikzcd}\]

\end{frame}

\begin{frame}{Prime Ideals in Cyclotomic Extensions of $\bbQ$}
Because the minimal polynomial of $\varepsilon - 1$ is a degree $p-1$ irreducible polynomial that is Eisenstein at $p$, we find that $\fp$ is fully ramified in the extension $\bbQ(\mu_p, \mu_{q-1}) / \bbQ(\mu_{q-1})$ (see \cite{Serre1979}) and so \[\fp \cO_{\bbQ(\mu_p,\mu_{q-1})} = \fP^{p-1}.\]

\pause
\vspace{0.5cm}
The minimal polynomial of $\varepsilon - 1$ is \[f(x) = \frac{(x+1)^p -1}{x}.\]
\end{frame}


\begin{frame}{The Teichm\"uller Character}
The \textbf{Teichm\"uller character} is a canonical example of a multiplicative character that generates $\widehat{\bbF_q^\times}$.\\
\pause
\vspace{0.5cm}

We have that $\cO_{\bbQ(\mu_{q-1})} / \fp$ is a field with multiplicative group isomorphic to $\mu_{q-1}$ which is in turn isomorphic to $\bbF_q^\times$. This gives us a unique character $\omega$ with $\omega(u) \equiv u \mod \fp$, for all units $u \in \bbF_q^\times$. We call $\omega$ the Teichm\"uller character.
\\
\pause
\vspace{0.5cm}
As mentioned, for all $\chi \in \widehat{\bbF_q^\times}$ we have that there is an integer $k$ with $0 \leq k < q-1$ such that $\chi = \omega^k$.

\end{frame}

\begin{frame}{Gauss Sums}
We define a \textbf{Gauss sum} of an additive character $\lambda$ and a multiplicative character $\chi$ as \[S(\chi, \lambda) = S(\chi) = \sum_{u \in \bbF_q^\times} \chi(u) \lambda(u).\]
\\
\pause
\vspace{0.5cm}

We will always fix the additive character $\lambda = \varepsilon^{\tr}$ and as mentioned before, in this case $S(\chi) \in \bbQ(\mu_p, \mu_{q-1})$.\\
\vspace{0.5cm}

For the rest of the talk, we will only use $S(\chi)$ to mean $S(\chi, \lambda)$
\end{frame}

\begin{frame}{Properties of Gauss Sums}
If $\chi_1 = 1$ is the trivial character on $\bbF_q^\times$, then $S(\chi_1) = -1$.\\
\vspace{0.5cm}
We have $S(\overline{\chi}) = \chi(-1)\overline{S(\chi)}$ and for $\chi \neq 1$, we have $S(\chi) S(\overline{\chi}) = \chi(-1) q$.
\\
\pause
\vspace{0.5cm}
This gives us a magnitude for non-trivial gauss sums: 
\[ |S(\chi)| = q^{1/2}.\]
\end{frame}


\begin{frame}{$p$-adic digits}
For any integer $k$ with $0 \leq k < q-1$, we can write \[k = k_0 + k_1 p + \cdots k_{n-1} p^{n-1}.\]
\pause
\vspace{0.5cm}
Then we define the $q-1$ periodic functions $s$ and $\gamma$ on the interval $0 \leq k < q-1$ by
\[s(k) = k_0 + k_1 + \cdots +k_{n-1} \]
\[\gamma(k) = k_0! k_1! \cdots k_{n-1}!.\]
\end{frame}

\begin{frame}{A First Approximation to Stickelberger's Theorem}
\begin{theorem}[Kummer's Approximation]
For any integer $k$, we have \[\frac{S(\omega^{-k})}{(\varepsilon - 1)^{s(k)}} \equiv \frac{-1}{\gamma(k)} \mod \fP.\]
So \[\ord_{\fP} \langle S(\omega^{-k} ) \rangle= s(k).\]
\end{theorem}

We will frequently use $\pi = \varepsilon - 1$ to simplify things.
\end{frame}

\begin{frame}{The Galois Group of $\bbQ(\mu_p, \mu_{q-1})$}
The Galois group $\gal(\bbQ(\mu_p, \mu_{q-1})/\bbQ)$ has $\gal(\bbQ(\mu_p)/\bbQ)$ and $\gal(\bbQ(\mu_{q-1})/\bbQ)$ as subgroups in a natural way; and these are the automorphisms that we find most useful for studying Gauss sums. 

Recalling that $\varepsilon = e^{2 \pi i /p}$ and let $\zeta = e^{2 \pi i/ (q-1)}$ we name the following automorphisms:\\
\pause
\vspace{0.5cm}

For $c$ coprime to $q-1$, we have $\sigma_{c,1}$ naturally included in $\gal(\bbQ(\mu_{q-1})/\bbQ)$.
\[\sigma_{c,1} : \zeta \mapsto \zeta^c \text{ and } \sigma_{c,1} \text{ is identity on } \mu_p\]
\pause
\vspace{0.5cm}
For $v$ coprime to $p$, we have $\sigma_{1,v}$ naturally included in $\gal(\bbQ(\mu_{p})/\bbQ)$.
\[\sigma_{1,v} : \varepsilon \mapsto \varepsilon^v \text{ and } \sigma_{1,v} \text{ is identity on } \mu_{q-1}\]
\end{frame}

\begin{frame}{More Properties of Gauss Sums }
Using properties of the Galois groups of $\bbQ(\mu_p, \mu_{q-1})$ and its subextensions, we have that \[S(\chi^p) = S(\chi)\] 
\pause
\vspace{0.5cm}

More generally, we have that \[\sigma_{c,1}S(\chi) = S(\chi^c) \text{ and } \sigma_{1,v} S(\chi) = \overline{\chi}(v)S(\chi)\]
\end{frame}

\begin{frame}{Kummer's Approximation}
\begin{theorem}[Kummer's Approximation]
For any integer $k$, we have \[\frac{S(\omega^{-k})}{(\pi)^{s(k)}} \equiv \frac{-1}{\gamma(k)} \mod \fP.\]
So \[\ord_{\fP} \langle S(\omega^{-k} ) \rangle= s(k).\]
\end{theorem}
\end{frame}

\begin{frame}{(sketch) Proof of Kummer's Approximation}
For $k =0$; $\omega^k = 1$ and we have $-1$ on both sides of the congruence.\\
\pause
\vspace{0.5cm}
For $k=1$; we have \begin{align*}
S(\omega^{-1}) &= \sum_u \omega(u)^{-1} \varepsilon^{\tr u} \\
&= \sum_u \omega(u)^{-1} (1 + \pi)^{\tr u} \\
&= \sum_u \omega(u)^{-1} ( 1 + \pi \tr u + O(\pi^2))
\end{align*}
\pause
\vspace{0.5cm}
So looking mod $\fP$ we find that \[\omega(u)^{-1} \tr u \equiv 1 + u^{p-1} + \cdots + u^{p^{n-1} -1} \mod \fP\]
\end{frame}

\begin{frame}{(sketch) Proof of Kummer's Approximation}
Summing the $\omega(u)^{-1} \tr u$ over $u$ and rearranging; we get \[ \frac{S(\omega^{-1})}{\pi} \equiv -1 \mod \fP\]\\
\pause
\vspace{0.5cm}

The rest of the proof proceeds by induction on $k$. 
\\
\pause
\vspace{0.5cm}
We split into an easy case where $p \mid k$ so $pk' = k$ with $k' < k$ a case we have dealt with already, and a hard case where $p \nmid k$.

For a full proof see Lang's \textit{Cyclotomic Fields} \cite{Lang1990}.

\end{frame}

\begin{frame}{Stickelberger's Theorem}
For  $k$ a positive integer, $G = \gal(\bbQ(\mu_{q-1})/\bbQ)$, and $\langle t \rangle$ the smallest postive real congruent to $t$ mod $\bbZ$; we define the \textbf{Stickelberger element} by \[\theta(k,\fp) = \sum_{c \in (\bbZ/(q-1)\bbZ)^\times} \left \langle \frac{kc}{q-1}\right \rangle \sigma_{c,1}^{-1} \in \bbQ[G]\]
\\
\pause
\vspace{0.5cm}
\begin{theorem}[Stickelberger's Theorem]
We have the factorization of the ideal generated by $S(\omega^{-k})$ as \[\langle S(\omega^{-k})\rangle = \fP^{(p-1)\theta(k, \fp)} = \fp^{\theta(k,\fp)}.\]
\end{theorem}
\end{frame}

\begin{frame}{A (sketch) Proof of Stickelberger's Theorem}
We have that \[\ord_{\sigma_{c,1}^{-1} \fP} \langle S(\omega^{-k})\rangle = \ord_{ \fP} \sigma_{c,1}\langle S(\omega^{-k})\rangle \] so $\ord_{\sigma_{c,1}^{-1} \fP} \langle S(\omega^{-k})\rangle = s(kc)$.\\

\pause The isotropy group of $\fp$ in $G$ is the powers of $\sigma_{p^i,1}$ for $i = 0, \ldots, n-1$. 
\\
\pause
\vspace{0.5cm}
So $\sigma_{c,1}^{-1} \fp$ has multiplicity $\sum_{i=0}^{n-1} \left\langle \frac{kcp^i}{q-1} \right\rangle$ in $\fp^{\theta(k)}$.
\\
\pause
\vspace{0.5cm}
It suffices to prove \[s(k) = (p-1)\sum_{i=0}^{n-1} \left \langle \frac{k p^i}{q-1} \right \rangle.\]
\end{frame}

\begin{frame}{An Application of Stickelberger's Theorem}
One application of of Stickelberger's theorem is as a step in proving a local converse theorem for finite fields.\\
\pause
\vspace{0.5cm}
Such a local converse theorem says roughly that if $\chi_1$ and $\chi_2$ are characters of $\bbF_q^\times$; and $S(\chi_1 \otimes \widetilde{\sigma}) = S(\chi_2 \otimes \widetilde{\sigma})$ for all characters $\sigma$ of $\bbF_p^\times$; then $\chi_1 = \chi_2^{p^i}$ for some $0 \leq p^i \leq p^{n-1}$. Here $\widetilde{\sigma} = \sigma \circ N_{\bbF_q / \bbF_p}$.
\\
\pause
\vspace{0.5cm}
Stickelberger's theorem lets us reduce this question about Gauss sums to a question about sums of digits; saying for instance that if \[s\left(\alpha + k\frac{q-1}{p-1}\right) = s\left(\beta + k \frac{q-1}{p-1}\right)\] for all $0 \leq k < q-1$ then $\alpha \equiv \beta^{p^i} \mod q-1$ for some $i$.
\end{frame}



\begin{frame}{Thank You!}
Thank you for coming to my talk! Many thanks to Professor Gil Moss for advising my thesis, which this presentation uses material from.\\
\medskip
References:
\medskip
    \bibliographystyle{plain}
\bibliography{thesisbibliography.bib}
\end{frame}



\end{document}
